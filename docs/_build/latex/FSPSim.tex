%% Generated by Sphinx.
\def\sphinxdocclass{report}
\documentclass[letterpaper,10pt,english]{sphinxmanual}
\ifdefined\pdfpxdimen
   \let\sphinxpxdimen\pdfpxdimen\else\newdimen\sphinxpxdimen
\fi \sphinxpxdimen=.75bp\relax
\ifdefined\pdfimageresolution
    \pdfimageresolution= \numexpr \dimexpr1in\relax/\sphinxpxdimen\relax
\fi
%% let collapsible pdf bookmarks panel have high depth per default
\PassOptionsToPackage{bookmarksdepth=5}{hyperref}
%% turn off hyperref patch of \index as sphinx.xdy xindy module takes care of
%% suitable \hyperpage mark-up, working around hyperref-xindy incompatibility
\PassOptionsToPackage{hyperindex=false}{hyperref}
%% memoir class requires extra handling
\makeatletter\@ifclassloaded{memoir}
{\ifdefined\memhyperindexfalse\memhyperindexfalse\fi}{}\makeatother

\PassOptionsToPackage{booktabs}{sphinx}
\PassOptionsToPackage{colorrows}{sphinx}

\PassOptionsToPackage{warn}{textcomp}

\catcode`^^^^00a0\active\protected\def^^^^00a0{\leavevmode\nobreak\ }
\usepackage{cmap}
\usepackage{fontspec}
\defaultfontfeatures[\rmfamily,\sffamily,\ttfamily]{}
\usepackage{amsmath,amssymb,amstext}
\usepackage{polyglossia}
\setmainlanguage{english}



\setmainfont{FreeSerif}[
  Extension      = .otf,
  UprightFont    = *,
  ItalicFont     = *Italic,
  BoldFont       = *Bold,
  BoldItalicFont = *BoldItalic
]
\setsansfont{FreeSans}[
  Extension      = .otf,
  UprightFont    = *,
  ItalicFont     = *Oblique,
  BoldFont       = *Bold,
  BoldItalicFont = *BoldOblique,
]
\setmonofont{FreeMono}[
  Extension      = .otf,
  UprightFont    = *,
  ItalicFont     = *Oblique,
  BoldFont       = *Bold,
  BoldItalicFont = *BoldOblique,
]



\usepackage[Bjarne]{fncychap}
\usepackage{sphinx}

\fvset{fontsize=\small}
\usepackage{geometry}


% Include hyperref last.
\usepackage{hyperref}
% Fix anchor placement for figures with captions.
\usepackage{hypcap}% it must be loaded after hyperref.
% Set up styles of URL: it should be placed after hyperref.
\urlstyle{same}

\addto\captionsenglish{\renewcommand{\contentsname}{Contents:}}

\usepackage{sphinxmessages}
\setcounter{tocdepth}{5}
\setcounter{secnumdepth}{5}


\title{FSPSim Documentation}
\date{Sep 13, 2023}
\release{0.1}
\author{Indigo Brownhall, 
Charles Constant, 
Santosh Bhattarai, 
Lee Devlin, 
Marek Ziebart}
\newcommand{\sphinxlogo}{\vbox{}}
\renewcommand{\releasename}{Release}
\makeindex
\begin{document}

\pagestyle{empty}
\sphinxmaketitle
\pagestyle{plain}
\sphinxtableofcontents
\pagestyle{normal}
\phantomsection\label{\detokenize{index::doc}}



\chapter{Motivation and Significance}
\label{\detokenize{index:motivation-and-significance}}
\sphinxAtStartPar
Human space endeavors have expanded dramatically over the past decades, with a significant increase in the number of satellites currently in orbit.
The rapid growth and potential impacts of satellite launches have led to the development of various models and the need for a more accessible and adaptable modeling framework.
The FSPSim tool addresses this need by offering a transparent, configurable, and open\sphinxhyphen{}source solution. This tool empowers stakeholders to comprehend and address the potential impacts of satellite mega\sphinxhyphen{}constellations.
FSPSim aims to make source\sphinxhyphen{}sink evolutionary models more accessible to a broader audience. It allows stakeholders to utilize modeling insights crafted by the community, promoting a more inclusive approach to understanding the effects of various decisions on the space environment.

\begin{figure}[htbp]
\centering
\capstart

\noindent\sphinxincludegraphics[width=0.800\linewidth]{{FSPSim_system_diag_v1}.png}
\caption{Overview of FSPSim’s architecture, showcasing how user inputs influence simulation parameters, the classes, and data sources they connect with, and the propagation mechanism.}\label{\detokenize{index:id1}}\end{figure}


\chapter{Software Description}
\label{\detokenize{index:software-description}}
\sphinxAtStartPar
FSPSim consists of various classes and modules designed to simulate the behavior and impact of space objects.
\begin{itemize}
\item {} 
\sphinxAtStartPar
\sphinxstylestrong{The SpaceCatalogue Class:} Manages all SpaceObject instances and fetches data from major space object catalogs.

\item {} 
\sphinxAtStartPar
\sphinxstylestrong{The SpaceObject Class:} Represents an individual object in space with detailed attributes.

\item {} 
\sphinxAtStartPar
\sphinxstylestrong{The LaunchModel Module:} Transforms user predictions into SpaceObjects.

\item {} 
\sphinxAtStartPar
\sphinxstylestrong{The Propagator Module:} Drives the propagation of all SpaceObjects within a given SpaceCatalog.

\end{itemize}


\chapter{Software Functionalities}
\label{\detokenize{index:software-functionalities}}
\sphinxAtStartPar
Users specify simulation parameters using a JSON file. The results are saved in a pickle file and can be further analyzed using popular data science libraries.


\chapter{Illustrative Examples}
\label{\detokenize{index:illustrative-examples}}
\sphinxAtStartPar
Based on provided input parameters, users can generate various visualizations to analyze the simulation data.

\begin{figure}[htbp]
\centering
\capstart

\noindent\sphinxincludegraphics[width=0.800\linewidth]{{fspsim_example}.png}
\caption{A sample altitude time\sphinxhyphen{}series plot generated from a simulation.}\label{\detokenize{index:id2}}\end{figure}


\chapter{Current and Future Developments}
\label{\detokenize{index:current-and-future-developments}}
\sphinxAtStartPar
Key areas of focus include refining the Launch Model and expanding the preliminary launch cost estimator.

\sphinxstepscope


\section{fspsim package}
\label{\detokenize{fspsim:fspsim-package}}\label{\detokenize{fspsim::doc}}

\subsection{Subpackages}
\label{\detokenize{fspsim:subpackages}}
\sphinxstepscope


\subsubsection{fspsim.utils package}
\label{\detokenize{fspsim.utils:fspsim-utils-package}}\label{\detokenize{fspsim.utils::doc}}

\paragraph{Submodules}
\label{\detokenize{fspsim.utils:submodules}}

\paragraph{fspsim.utils.Conversions module}
\label{\detokenize{fspsim.utils:module-fspsim.utils.Conversions}}\label{\detokenize{fspsim.utils:fspsim-utils-conversions-module}}\index{module@\spxentry{module}!fspsim.utils.Conversions@\spxentry{fspsim.utils.Conversions}}\index{fspsim.utils.Conversions@\spxentry{fspsim.utils.Conversions}!module@\spxentry{module}}\index{TLE\_time() (in module fspsim.utils.Conversions)@\spxentry{TLE\_time()}\spxextra{in module fspsim.utils.Conversions}}

\begin{fulllineitems}
\phantomsection\label{\detokenize{fspsim.utils:fspsim.utils.Conversions.TLE_time}}
\pysigstartsignatures
\pysiglinewithargsret{\sphinxcode{\sphinxupquote{fspsim.utils.Conversions.}}\sphinxbfcode{\sphinxupquote{TLE\_time}}}{\sphinxparam{\DUrole{n}{TLE}}}{}
\pysigstopsignatures
\sphinxAtStartPar
Returns the Epoch Year of the TLE as a Julian Date.
\begin{quote}\begin{description}
\sphinxlineitem{Parameters}
\sphinxAtStartPar
\sphinxstyleliteralstrong{\sphinxupquote{TLE}} (\sphinxstyleliteralemphasis{\sphinxupquote{string}}) – Two Line Element

\sphinxlineitem{Returns}
\sphinxAtStartPar
Julian Date

\sphinxlineitem{Return type}
\sphinxAtStartPar
datetime.datetime

\end{description}\end{quote}

\end{fulllineitems}

\index{UTC\_step() (in module fspsim.utils.Conversions)@\spxentry{UTC\_step()}\spxextra{in module fspsim.utils.Conversions}}

\begin{fulllineitems}
\phantomsection\label{\detokenize{fspsim.utils:fspsim.utils.Conversions.UTC_step}}
\pysigstartsignatures
\pysiglinewithargsret{\sphinxcode{\sphinxupquote{fspsim.utils.Conversions.}}\sphinxbfcode{\sphinxupquote{UTC\_step}}}{\sphinxparam{\DUrole{n}{date\_list}}\sphinxparamcomma \sphinxparam{\DUrole{n}{steps}}\sphinxparamcomma \sphinxparam{\DUrole{n}{h}}}{}
\pysigstopsignatures
\sphinxAtStartPar
Make an array of datetime strings in UTC format.
\begin{quote}\begin{description}
\sphinxlineitem{Parameters}\begin{itemize}
\item {} 
\sphinxAtStartPar
\sphinxstyleliteralstrong{\sphinxupquote{date\_list}} (\sphinxstyleliteralemphasis{\sphinxupquote{list}}\sphinxstyleliteralemphasis{\sphinxupquote{{[}}}\sphinxstyleliteralemphasis{\sphinxupquote{int}}\sphinxstyleliteralemphasis{\sphinxupquote{{]}}}) – List containing year, month, day, hour, minute, and second.

\item {} 
\sphinxAtStartPar
\sphinxstyleliteralstrong{\sphinxupquote{steps}} (\sphinxstyleliteralemphasis{\sphinxupquote{int}}) – Number of steps in the propagation.

\item {} 
\sphinxAtStartPar
\sphinxstyleliteralstrong{\sphinxupquote{h}} (\sphinxstyleliteralemphasis{\sphinxupquote{float}}) – Step size the propagation uses in seconds.

\end{itemize}

\sphinxlineitem{Returns}
\sphinxAtStartPar
List of datetime strings in UTC format.

\sphinxlineitem{Return type}
\sphinxAtStartPar
list{[}datetime.datetime{]}

\end{description}\end{quote}

\end{fulllineitems}

\index{calculate\_eccentricity() (in module fspsim.utils.Conversions)@\spxentry{calculate\_eccentricity()}\spxextra{in module fspsim.utils.Conversions}}

\begin{fulllineitems}
\phantomsection\label{\detokenize{fspsim.utils:fspsim.utils.Conversions.calculate_eccentricity}}
\pysigstartsignatures
\pysiglinewithargsret{\sphinxcode{\sphinxupquote{fspsim.utils.Conversions.}}\sphinxbfcode{\sphinxupquote{calculate\_eccentricity}}}{\sphinxparam{\DUrole{n}{position}\DUrole{p}{:}\DUrole{w}{ }\DUrole{n}{List\DUrole{p}{{[}}float\DUrole{p}{{]}}}}\sphinxparamcomma \sphinxparam{\DUrole{n}{velocity}\DUrole{p}{:}\DUrole{w}{ }\DUrole{n}{List\DUrole{p}{{[}}float\DUrole{p}{{]}}}}\sphinxparamcomma \sphinxparam{\DUrole{n}{mu}\DUrole{p}{:}\DUrole{w}{ }\DUrole{n}{float}}}{}
\pysigstopsignatures
\sphinxAtStartPar
Calculate the eccentricity of an orbit given position, velocity, and gravitational parameter.

\sphinxAtStartPar
The function computes the eccentricity using the specific mechanical energy and specific angular momentum
of the object in orbit.
\begin{quote}\begin{description}
\sphinxlineitem{Parameters}\begin{itemize}
\item {} 
\sphinxAtStartPar
\sphinxstyleliteralstrong{\sphinxupquote{position}} (\sphinxstyleliteralemphasis{\sphinxupquote{List}}\sphinxstyleliteralemphasis{\sphinxupquote{{[}}}\sphinxstyleliteralemphasis{\sphinxupquote{float}}\sphinxstyleliteralemphasis{\sphinxupquote{{]}}}) – Position vector of the object in orbit. Expected to be a 3\sphinxhyphen{}element list representing {[}x, y, z{]}.

\item {} 
\sphinxAtStartPar
\sphinxstyleliteralstrong{\sphinxupquote{velocity}} (\sphinxstyleliteralemphasis{\sphinxupquote{List}}\sphinxstyleliteralemphasis{\sphinxupquote{{[}}}\sphinxstyleliteralemphasis{\sphinxupquote{float}}\sphinxstyleliteralemphasis{\sphinxupquote{{]}}}) – Velocity vector of the object in orbit. Expected to be a 3\sphinxhyphen{}element list representing {[}vx, vy, vz{]}.

\item {} 
\sphinxAtStartPar
\sphinxstyleliteralstrong{\sphinxupquote{mu}} (\sphinxstyleliteralemphasis{\sphinxupquote{float}}) – Gravitational parameter, product of the gravitational constant (G) and the mass (M) of the primary body.

\end{itemize}

\sphinxlineitem{Returns}
\sphinxAtStartPar
Eccentricity of the orbit.

\sphinxlineitem{Return type}
\sphinxAtStartPar
float

\end{description}\end{quote}

\end{fulllineitems}

\index{calculate\_energy\_from\_semi\_major\_axis() (in module fspsim.utils.Conversions)@\spxentry{calculate\_energy\_from\_semi\_major\_axis()}\spxextra{in module fspsim.utils.Conversions}}

\begin{fulllineitems}
\phantomsection\label{\detokenize{fspsim.utils:fspsim.utils.Conversions.calculate_energy_from_semi_major_axis}}
\pysigstartsignatures
\pysiglinewithargsret{\sphinxcode{\sphinxupquote{fspsim.utils.Conversions.}}\sphinxbfcode{\sphinxupquote{calculate\_energy\_from\_semi\_major\_axis}}}{\sphinxparam{\DUrole{n}{a}}\sphinxparamcomma \sphinxparam{\DUrole{n}{m}}}{}
\pysigstopsignatures
\sphinxAtStartPar
Use the vis\sphinxhyphen{}viva law to calculate the total mechanical energy of an object in orbit around the Earth.
\begin{quote}\begin{description}
\sphinxlineitem{Parameters}\begin{itemize}
\item {} 
\sphinxAtStartPar
\sphinxstyleliteralstrong{\sphinxupquote{a}} (\sphinxstyleliteralemphasis{\sphinxupquote{float}}) – Semi\sphinxhyphen{}major axis in meters.

\item {} 
\sphinxAtStartPar
\sphinxstyleliteralstrong{\sphinxupquote{m}} (\sphinxstyleliteralemphasis{\sphinxupquote{float}}) – Mass of the object in kg.

\end{itemize}

\sphinxlineitem{Returns}
\sphinxAtStartPar
Total mechanical energy in joules.

\sphinxlineitem{Return type}
\sphinxAtStartPar
float

\end{description}\end{quote}

\end{fulllineitems}

\index{car2kep() (in module fspsim.utils.Conversions)@\spxentry{car2kep()}\spxextra{in module fspsim.utils.Conversions}}

\begin{fulllineitems}
\phantomsection\label{\detokenize{fspsim.utils:fspsim.utils.Conversions.car2kep}}
\pysigstartsignatures
\pysiglinewithargsret{\sphinxcode{\sphinxupquote{fspsim.utils.Conversions.}}\sphinxbfcode{\sphinxupquote{car2kep}}}{\sphinxparam{\DUrole{n}{x}}\sphinxparamcomma \sphinxparam{\DUrole{n}{y}}\sphinxparamcomma \sphinxparam{\DUrole{n}{z}}\sphinxparamcomma \sphinxparam{\DUrole{n}{vx}}\sphinxparamcomma \sphinxparam{\DUrole{n}{vy}}\sphinxparamcomma \sphinxparam{\DUrole{n}{vz}}\sphinxparamcomma \sphinxparam{\DUrole{n}{mean\_motion}\DUrole{o}{=}\DUrole{default_value}{False}}\sphinxparamcomma \sphinxparam{\DUrole{n}{arg\_l}\DUrole{o}{=}\DUrole{default_value}{False}}}{}
\pysigstopsignatures
\sphinxAtStartPar
Convert Cartesian coordinates to Keplerian elements in radians.
\begin{quote}\begin{description}
\sphinxlineitem{Parameters}\begin{itemize}
\item {} 
\sphinxAtStartPar
\sphinxstyleliteralstrong{\sphinxupquote{x}} (\sphinxstyleliteralemphasis{\sphinxupquote{float}}) – Position coordinate in km.

\item {} 
\sphinxAtStartPar
\sphinxstyleliteralstrong{\sphinxupquote{y}} (\sphinxstyleliteralemphasis{\sphinxupquote{float}}) – Position coordinate in km.

\item {} 
\sphinxAtStartPar
\sphinxstyleliteralstrong{\sphinxupquote{z}} (\sphinxstyleliteralemphasis{\sphinxupquote{float}}) – Position coordinate in km.

\item {} 
\sphinxAtStartPar
\sphinxstyleliteralstrong{\sphinxupquote{vx}} (\sphinxstyleliteralemphasis{\sphinxupquote{float}}) – Velocity component in km/s.

\item {} 
\sphinxAtStartPar
\sphinxstyleliteralstrong{\sphinxupquote{vy}} (\sphinxstyleliteralemphasis{\sphinxupquote{float}}) – Velocity component in km/s.

\item {} 
\sphinxAtStartPar
\sphinxstyleliteralstrong{\sphinxupquote{vz}} (\sphinxstyleliteralemphasis{\sphinxupquote{float}}) – Velocity component in km/s.

\item {} 
\sphinxAtStartPar
\sphinxstyleliteralstrong{\sphinxupquote{mean\_motion}} (\sphinxstyleliteralemphasis{\sphinxupquote{bool}}) – If True, return mean motion. Default is False.

\item {} 
\sphinxAtStartPar
\sphinxstyleliteralstrong{\sphinxupquote{arg\_l}} (\sphinxstyleliteralemphasis{\sphinxupquote{bool}}) – If True, return the argument of latitude. Default is False.

\end{itemize}

\sphinxlineitem{Returns}
\sphinxAtStartPar
Tuple containing semi\sphinxhyphen{}major axis, eccentricity, inclination,
RAAN, argument of perigee, and true anomaly. Optionally returns
mean motion and argument of latitude.

\sphinxlineitem{Return type}
\sphinxAtStartPar
tuple

\end{description}\end{quote}

\end{fulllineitems}

\index{earth\_moon\_vec() (in module fspsim.utils.Conversions)@\spxentry{earth\_moon\_vec()}\spxextra{in module fspsim.utils.Conversions}}

\begin{fulllineitems}
\phantomsection\label{\detokenize{fspsim.utils:fspsim.utils.Conversions.earth_moon_vec}}
\pysigstartsignatures
\pysiglinewithargsret{\sphinxcode{\sphinxupquote{fspsim.utils.Conversions.}}\sphinxbfcode{\sphinxupquote{earth\_moon\_vec}}}{\sphinxparam{\DUrole{n}{jd}}\sphinxparamcomma \sphinxparam{\DUrole{n}{unit}\DUrole{o}{=}\DUrole{default_value}{True}}}{}
\pysigstopsignatures
\sphinxAtStartPar
Calculate the Earth\sphinxhyphen{}Moon vector in ECI coordinates.
Makes use of the JPL ephemerides to calculate the Earth\sphinxhyphen{}Moon vector.
\begin{quote}\begin{description}
\sphinxlineitem{Parameters}\begin{itemize}
\item {} 
\sphinxAtStartPar
\sphinxstyleliteralstrong{\sphinxupquote{jd}} (\sphinxstyleliteralemphasis{\sphinxupquote{float}}) – Julian Date for which we want the Earth\sphinxhyphen{}Moon vector.

\item {} 
\sphinxAtStartPar
\sphinxstyleliteralstrong{\sphinxupquote{unit}} (\sphinxstyleliteralemphasis{\sphinxupquote{bool}}) – If True, returns the unit vector. If False, returns the vector itself. Default is True.

\end{itemize}

\sphinxlineitem{Returns}
\sphinxAtStartPar
Earth\sphinxhyphen{}Moon vector in ECI coordinates.

\sphinxlineitem{Return type}
\sphinxAtStartPar
np.array

\end{description}\end{quote}

\end{fulllineitems}

\index{earth\_sun\_vec() (in module fspsim.utils.Conversions)@\spxentry{earth\_sun\_vec()}\spxextra{in module fspsim.utils.Conversions}}

\begin{fulllineitems}
\phantomsection\label{\detokenize{fspsim.utils:fspsim.utils.Conversions.earth_sun_vec}}
\pysigstartsignatures
\pysiglinewithargsret{\sphinxcode{\sphinxupquote{fspsim.utils.Conversions.}}\sphinxbfcode{\sphinxupquote{earth\_sun\_vec}}}{\sphinxparam{\DUrole{n}{jd}}\sphinxparamcomma \sphinxparam{\DUrole{n}{unit}\DUrole{o}{=}\DUrole{default_value}{True}}}{}
\pysigstopsignatures
\sphinxAtStartPar
Calculate the Earth\sphinxhyphen{}Sun vector in ECI coordinates.
Makes use of the JPL ephemerides to calculate the Earth\sphinxhyphen{}Sun vector.
\begin{quote}\begin{description}
\sphinxlineitem{Parameters}\begin{itemize}
\item {} 
\sphinxAtStartPar
\sphinxstyleliteralstrong{\sphinxupquote{jd}} (\sphinxstyleliteralemphasis{\sphinxupquote{float}}) – Julian Date for which we want the Earth\sphinxhyphen{}Sun vector.

\item {} 
\sphinxAtStartPar
\sphinxstyleliteralstrong{\sphinxupquote{unit}} (\sphinxstyleliteralemphasis{\sphinxupquote{bool}}) – If True, returns the unit vector. If False, returns the vector itself. Default is True.

\end{itemize}

\sphinxlineitem{Returns}
\sphinxAtStartPar
Earth\sphinxhyphen{}Sun vector in ECI coordinates.

\sphinxlineitem{Return type}
\sphinxAtStartPar
np.array

\end{description}\end{quote}

\end{fulllineitems}

\index{eccentric\_to\_mean\_anomaly() (in module fspsim.utils.Conversions)@\spxentry{eccentric\_to\_mean\_anomaly()}\spxextra{in module fspsim.utils.Conversions}}

\begin{fulllineitems}
\phantomsection\label{\detokenize{fspsim.utils:fspsim.utils.Conversions.eccentric_to_mean_anomaly}}
\pysigstartsignatures
\pysiglinewithargsret{\sphinxcode{\sphinxupquote{fspsim.utils.Conversions.}}\sphinxbfcode{\sphinxupquote{eccentric\_to\_mean\_anomaly}}}{\sphinxparam{\DUrole{n}{eccentric\_anomaly}}\sphinxparamcomma \sphinxparam{\DUrole{n}{eccentricity}}}{}
\pysigstopsignatures
\sphinxAtStartPar
Convert eccentric anomaly to mean anomaly.
\begin{quote}\begin{description}
\sphinxlineitem{Parameters}\begin{itemize}
\item {} 
\sphinxAtStartPar
\sphinxstyleliteralstrong{\sphinxupquote{eccentric\_anomaly}} (\sphinxstyleliteralemphasis{\sphinxupquote{float}}) – Eccentric anomaly in radians.

\item {} 
\sphinxAtStartPar
\sphinxstyleliteralstrong{\sphinxupquote{eccentricity}} (\sphinxstyleliteralemphasis{\sphinxupquote{float}}) – Orbital eccentricity.

\end{itemize}

\sphinxlineitem{Returns}
\sphinxAtStartPar
Mean anomaly in radians.

\sphinxlineitem{Return type}
\sphinxAtStartPar
float

\end{description}\end{quote}

\end{fulllineitems}

\index{ecef2eci\_astropy() (in module fspsim.utils.Conversions)@\spxentry{ecef2eci\_astropy()}\spxextra{in module fspsim.utils.Conversions}}

\begin{fulllineitems}
\phantomsection\label{\detokenize{fspsim.utils:fspsim.utils.Conversions.ecef2eci_astropy}}
\pysigstartsignatures
\pysiglinewithargsret{\sphinxcode{\sphinxupquote{fspsim.utils.Conversions.}}\sphinxbfcode{\sphinxupquote{ecef2eci\_astropy}}}{\sphinxparam{\DUrole{n}{ecef\_pos}\DUrole{p}{:}\DUrole{w}{ }\DUrole{n}{ndarray}}\sphinxparamcomma \sphinxparam{\DUrole{n}{ecef\_vel}\DUrole{p}{:}\DUrole{w}{ }\DUrole{n}{ndarray}}\sphinxparamcomma \sphinxparam{\DUrole{n}{mjd}\DUrole{p}{:}\DUrole{w}{ }\DUrole{n}{float}}}{{ $\rightarrow$ Tuple\DUrole{p}{{[}}ndarray\DUrole{p}{,}\DUrole{w}{ }ndarray\DUrole{p}{{]}}}}
\pysigstopsignatures
\sphinxAtStartPar
Convert ECEF (Earth\sphinxhyphen{}Centered, Earth\sphinxhyphen{}Fixed) coordinates to ECI (Earth\sphinxhyphen{}Centered Inertial) coordinates using Astropy.
\begin{quote}\begin{description}
\sphinxlineitem{Parameters}\begin{itemize}
\item {} 
\sphinxAtStartPar
\sphinxstyleliteralstrong{\sphinxupquote{ecef\_pos}} (\sphinxstyleliteralemphasis{\sphinxupquote{np.ndarray}}) – ECEF position vectors.

\item {} 
\sphinxAtStartPar
\sphinxstyleliteralstrong{\sphinxupquote{ecef\_vel}} (\sphinxstyleliteralemphasis{\sphinxupquote{np.ndarray}}) – ECEF velocity vectors.

\item {} 
\sphinxAtStartPar
\sphinxstyleliteralstrong{\sphinxupquote{mjd}} (\sphinxstyleliteralemphasis{\sphinxupquote{float}}) – Modified Julian Date.

\end{itemize}

\sphinxlineitem{Returns}
\sphinxAtStartPar
Tuple containing ECI position and ECI velocity vectors.

\sphinxlineitem{Return type}
\sphinxAtStartPar
tuple of np.ndarray

\end{description}\end{quote}

\end{fulllineitems}

\index{generate\_cospar\_id() (in module fspsim.utils.Conversions)@\spxentry{generate\_cospar\_id()}\spxextra{in module fspsim.utils.Conversions}}

\begin{fulllineitems}
\phantomsection\label{\detokenize{fspsim.utils:fspsim.utils.Conversions.generate_cospar_id}}
\pysigstartsignatures
\pysiglinewithargsret{\sphinxcode{\sphinxupquote{fspsim.utils.Conversions.}}\sphinxbfcode{\sphinxupquote{generate\_cospar\_id}}}{\sphinxparam{\DUrole{n}{launch\_year}}\sphinxparamcomma \sphinxparam{\DUrole{n}{launch\_number}}\sphinxparamcomma \sphinxparam{\DUrole{n}{launch\_piece}}}{}
\pysigstopsignatures
\sphinxAtStartPar
Generates a placeholder (fake) COSPAR ID for a spacecraft based on the launch year, launch number, and launch piece.
\begin{quote}\begin{description}
\sphinxlineitem{Parameters}\begin{itemize}
\item {} 
\sphinxAtStartPar
\sphinxstyleliteralstrong{\sphinxupquote{launch\_year}} (\sphinxstyleliteralemphasis{\sphinxupquote{int}}) – The launch year of the spacecraft.

\item {} 
\sphinxAtStartPar
\sphinxstyleliteralstrong{\sphinxupquote{launch\_number}} (\sphinxstyleliteralemphasis{\sphinxupquote{int}}) – The launch number of the spacecraft.

\item {} 
\sphinxAtStartPar
\sphinxstyleliteralstrong{\sphinxupquote{launch\_piece}} (\sphinxstyleliteralemphasis{\sphinxupquote{str}}) – The piece of the launch.

\end{itemize}

\sphinxlineitem{Returns}
\sphinxAtStartPar
The generated COSPAR ID.

\sphinxlineitem{Return type}
\sphinxAtStartPar
str

\end{description}\end{quote}

\end{fulllineitems}

\index{get\_day\_of\_year\_and\_fractional\_day() (in module fspsim.utils.Conversions)@\spxentry{get\_day\_of\_year\_and\_fractional\_day()}\spxextra{in module fspsim.utils.Conversions}}

\begin{fulllineitems}
\phantomsection\label{\detokenize{fspsim.utils:fspsim.utils.Conversions.get_day_of_year_and_fractional_day}}
\pysigstartsignatures
\pysiglinewithargsret{\sphinxcode{\sphinxupquote{fspsim.utils.Conversions.}}\sphinxbfcode{\sphinxupquote{get\_day\_of\_year\_and\_fractional\_day}}}{\sphinxparam{\DUrole{n}{epoch}}}{}
\pysigstopsignatures
\sphinxAtStartPar
Compute the day of the year and fractional day for a given date.
\begin{quote}\begin{description}
\sphinxlineitem{Parameters}
\sphinxAtStartPar
\sphinxstyleliteralstrong{\sphinxupquote{epoch}} (\sphinxstyleliteralemphasis{\sphinxupquote{datetime.datetime}}) – Datetime object representing the epoch.

\sphinxlineitem{Returns}
\sphinxAtStartPar
Day of the year and fractional day.

\sphinxlineitem{Return type}
\sphinxAtStartPar
float

\end{description}\end{quote}

\end{fulllineitems}

\index{jd\_to\_utc() (in module fspsim.utils.Conversions)@\spxentry{jd\_to\_utc()}\spxextra{in module fspsim.utils.Conversions}}

\begin{fulllineitems}
\phantomsection\label{\detokenize{fspsim.utils:fspsim.utils.Conversions.jd_to_utc}}
\pysigstartsignatures
\pysiglinewithargsret{\sphinxcode{\sphinxupquote{fspsim.utils.Conversions.}}\sphinxbfcode{\sphinxupquote{jd\_to\_utc}}}{\sphinxparam{\DUrole{n}{jd}}}{}
\pysigstopsignatures
\sphinxAtStartPar
Converts Julian Date to UTC time tag (datetime object) using Astropy.
\begin{quote}\begin{description}
\sphinxlineitem{Parameters}
\sphinxAtStartPar
\sphinxstyleliteralstrong{\sphinxupquote{jd}} (\sphinxstyleliteralemphasis{\sphinxupquote{float}}) – Julian Date

\sphinxlineitem{Returns}
\sphinxAtStartPar
UTC time tag (datetime object)

\sphinxlineitem{Return type}
\sphinxAtStartPar
datetime.datetime

\end{description}\end{quote}

\end{fulllineitems}

\index{kep2car() (in module fspsim.utils.Conversions)@\spxentry{kep2car()}\spxextra{in module fspsim.utils.Conversions}}

\begin{fulllineitems}
\phantomsection\label{\detokenize{fspsim.utils:fspsim.utils.Conversions.kep2car}}
\pysigstartsignatures
\pysiglinewithargsret{\sphinxcode{\sphinxupquote{fspsim.utils.Conversions.}}\sphinxbfcode{\sphinxupquote{kep2car}}}{\sphinxparam{\DUrole{n}{a}}\sphinxparamcomma \sphinxparam{\DUrole{n}{e}}\sphinxparamcomma \sphinxparam{\DUrole{n}{i}}\sphinxparamcomma \sphinxparam{\DUrole{n}{w}}\sphinxparamcomma \sphinxparam{\DUrole{n}{W}}\sphinxparamcomma \sphinxparam{\DUrole{n}{V}}\sphinxparamcomma \sphinxparam{\DUrole{n}{epoch}}}{}
\pysigstopsignatures
\sphinxAtStartPar
Convert Keplerian elements to Cartesian coordinates.
\begin{quote}\begin{description}
\sphinxlineitem{Parameters}\begin{itemize}
\item {} 
\sphinxAtStartPar
\sphinxstyleliteralstrong{\sphinxupquote{a}} – Semi\sphinxhyphen{}major axis (km).

\item {} 
\sphinxAtStartPar
\sphinxstyleliteralstrong{\sphinxupquote{e}} – Eccentricity.

\item {} 
\sphinxAtStartPar
\sphinxstyleliteralstrong{\sphinxupquote{i}} – Inclination (rad).

\item {} 
\sphinxAtStartPar
\sphinxstyleliteralstrong{\sphinxupquote{w}} – Argument of perigee (rad).

\item {} 
\sphinxAtStartPar
\sphinxstyleliteralstrong{\sphinxupquote{W}} – Right ascension of ascending node (RAAN) (rad).

\item {} 
\sphinxAtStartPar
\sphinxstyleliteralstrong{\sphinxupquote{V}} – True anomaly (rad).

\item {} 
\sphinxAtStartPar
\sphinxstyleliteralstrong{\sphinxupquote{epoch}} (\sphinxstyleliteralemphasis{\sphinxupquote{Time}}) – Time at which the elements are given.

\end{itemize}

\sphinxlineitem{Returns}
\sphinxAtStartPar
Tuple containing x, y, z coordinates and vx, vy, vz velocities.

\sphinxlineitem{Return type}
\sphinxAtStartPar
tuple of floats

\end{description}\end{quote}

\end{fulllineitems}

\index{orbit\_classify() (in module fspsim.utils.Conversions)@\spxentry{orbit\_classify()}\spxextra{in module fspsim.utils.Conversions}}

\begin{fulllineitems}
\phantomsection\label{\detokenize{fspsim.utils:fspsim.utils.Conversions.orbit_classify}}
\pysigstartsignatures
\pysiglinewithargsret{\sphinxcode{\sphinxupquote{fspsim.utils.Conversions.}}\sphinxbfcode{\sphinxupquote{orbit\_classify}}}{\sphinxparam{\DUrole{n}{altitude}}}{}
\pysigstopsignatures
\sphinxAtStartPar
Classify the orbit based on the altitude.
\begin{quote}\begin{description}
\sphinxlineitem{Parameters}
\sphinxAtStartPar
\sphinxstyleliteralstrong{\sphinxupquote{altitude}} (\sphinxstyleliteralemphasis{\sphinxupquote{float}}) – Altitude of the orbit in km.

\sphinxlineitem{Returns}
\sphinxAtStartPar
Orbit classification as a string.

\sphinxlineitem{Return type}
\sphinxAtStartPar
str

\end{description}\end{quote}

\end{fulllineitems}

\index{orbital\_period() (in module fspsim.utils.Conversions)@\spxentry{orbital\_period()}\spxextra{in module fspsim.utils.Conversions}}

\begin{fulllineitems}
\phantomsection\label{\detokenize{fspsim.utils:fspsim.utils.Conversions.orbital_period}}
\pysigstartsignatures
\pysiglinewithargsret{\sphinxcode{\sphinxupquote{fspsim.utils.Conversions.}}\sphinxbfcode{\sphinxupquote{orbital\_period}}}{\sphinxparam{\DUrole{n}{semi\_major\_axis}}}{}
\pysigstopsignatures
\sphinxAtStartPar
Calculate the Orbital Period of a satellite based on the semi\sphinxhyphen{}major\sphinxhyphen{}axis
\begin{quote}\begin{description}
\sphinxlineitem{Parameters}
\sphinxAtStartPar
\sphinxstyleliteralstrong{\sphinxupquote{semi\_major\_axis}} (\sphinxstyleliteralemphasis{\sphinxupquote{int}}) – Semi Major Axis

\sphinxlineitem{Returns}
\sphinxAtStartPar
Orbital Period in Minutes

\sphinxlineitem{Return type}
\sphinxAtStartPar
int

\end{description}\end{quote}

\end{fulllineitems}

\index{probe\_sun\_vec() (in module fspsim.utils.Conversions)@\spxentry{probe\_sun\_vec()}\spxextra{in module fspsim.utils.Conversions}}

\begin{fulllineitems}
\phantomsection\label{\detokenize{fspsim.utils:fspsim.utils.Conversions.probe_sun_vec}}
\pysigstartsignatures
\pysiglinewithargsret{\sphinxcode{\sphinxupquote{fspsim.utils.Conversions.}}\sphinxbfcode{\sphinxupquote{probe\_sun\_vec}}}{\sphinxparam{\DUrole{n}{r}}\sphinxparamcomma \sphinxparam{\DUrole{n}{jd}}\sphinxparamcomma \sphinxparam{\DUrole{n}{unit}\DUrole{o}{=}\DUrole{default_value}{False}}}{}
\pysigstopsignatures
\sphinxAtStartPar
Calculate the probe\sphinxhyphen{}Sun vector in ECI coordinates.
\begin{quote}\begin{description}
\sphinxlineitem{Parameters}\begin{itemize}
\item {} 
\sphinxAtStartPar
\sphinxstyleliteralstrong{\sphinxupquote{r}} (\sphinxstyleliteralemphasis{\sphinxupquote{array}}) – Probe position in ECI coordinates.

\item {} 
\sphinxAtStartPar
\sphinxstyleliteralstrong{\sphinxupquote{jd}} (\sphinxstyleliteralemphasis{\sphinxupquote{float}}) – Julian Date for which we want the probe\sphinxhyphen{}Sun vector.

\item {} 
\sphinxAtStartPar
\sphinxstyleliteralstrong{\sphinxupquote{unit}} (\sphinxstyleliteralemphasis{\sphinxupquote{bool}}) – If True, returns the unit vector. If False, returns the vector itself. Default is False.

\end{itemize}

\sphinxlineitem{Returns}
\sphinxAtStartPar
Probe\sphinxhyphen{}Sun vector in ECI coordinates.

\sphinxlineitem{Return type}
\sphinxAtStartPar
np.array

\end{description}\end{quote}

\end{fulllineitems}

\index{tle\_checksum() (in module fspsim.utils.Conversions)@\spxentry{tle\_checksum()}\spxextra{in module fspsim.utils.Conversions}}

\begin{fulllineitems}
\phantomsection\label{\detokenize{fspsim.utils:fspsim.utils.Conversions.tle_checksum}}
\pysigstartsignatures
\pysiglinewithargsret{\sphinxcode{\sphinxupquote{fspsim.utils.Conversions.}}\sphinxbfcode{\sphinxupquote{tle\_checksum}}}{\sphinxparam{\DUrole{n}{line}}}{}
\pysigstopsignatures
\sphinxAtStartPar
Perform the checksum for one TLE line
\begin{quote}\begin{description}
\sphinxlineitem{Parameters}
\sphinxAtStartPar
\sphinxstyleliteralstrong{\sphinxupquote{line}} (\sphinxstyleliteralemphasis{\sphinxupquote{str}}) – TLE line

\sphinxlineitem{Returns}
\sphinxAtStartPar
Checksum

\sphinxlineitem{Return type}
\sphinxAtStartPar
int

\end{description}\end{quote}

\end{fulllineitems}

\index{tle\_convert() (in module fspsim.utils.Conversions)@\spxentry{tle\_convert()}\spxextra{in module fspsim.utils.Conversions}}

\begin{fulllineitems}
\phantomsection\label{\detokenize{fspsim.utils:fspsim.utils.Conversions.tle_convert}}
\pysigstartsignatures
\pysiglinewithargsret{\sphinxcode{\sphinxupquote{fspsim.utils.Conversions.}}\sphinxbfcode{\sphinxupquote{tle\_convert}}}{\sphinxparam{\DUrole{n}{tle\_dict}}\sphinxparamcomma \sphinxparam{\DUrole{n}{display}\DUrole{o}{=}\DUrole{default_value}{False}}}{}
\pysigstopsignatures
\sphinxAtStartPar
Convert a TLE dictionary into the corresponding Keplerian elements.
\begin{quote}\begin{description}
\sphinxlineitem{Parameters}\begin{itemize}
\item {} 
\sphinxAtStartPar
\sphinxstyleliteralstrong{\sphinxupquote{tle\_dict}} (\sphinxstyleliteralemphasis{\sphinxupquote{dict}}) – Dictionary of TLE data as provided by the tle\_parse function.

\item {} 
\sphinxAtStartPar
\sphinxstyleliteralstrong{\sphinxupquote{display}} (\sphinxstyleliteralemphasis{\sphinxupquote{bool}}) – If True, print out the Keplerian elements. Default is False.

\end{itemize}

\sphinxlineitem{Returns}
\sphinxAtStartPar
Dictionary containing Keplerian elements.

\sphinxlineitem{Return type}
\sphinxAtStartPar
dict

\end{description}\end{quote}

\end{fulllineitems}

\index{tle\_exponent\_format() (in module fspsim.utils.Conversions)@\spxentry{tle\_exponent\_format()}\spxextra{in module fspsim.utils.Conversions}}

\begin{fulllineitems}
\phantomsection\label{\detokenize{fspsim.utils:fspsim.utils.Conversions.tle_exponent_format}}
\pysigstartsignatures
\pysiglinewithargsret{\sphinxcode{\sphinxupquote{fspsim.utils.Conversions.}}\sphinxbfcode{\sphinxupquote{tle\_exponent\_format}}}{\sphinxparam{\DUrole{n}{value}}}{}
\pysigstopsignatures
\sphinxAtStartPar
Convert a value to the TLE\sphinxhyphen{}specific exponent format.
\begin{quote}\begin{description}
\sphinxlineitem{Parameters}
\sphinxAtStartPar
\sphinxstyleliteralstrong{\sphinxupquote{value}} (\sphinxstyleliteralemphasis{\sphinxupquote{str}}) – TLE Value

\sphinxlineitem{Returns}
\sphinxAtStartPar
Formatted value

\sphinxlineitem{Return type}
\sphinxAtStartPar
str

\end{description}\end{quote}

\end{fulllineitems}

\index{tle\_parse() (in module fspsim.utils.Conversions)@\spxentry{tle\_parse()}\spxextra{in module fspsim.utils.Conversions}}

\begin{fulllineitems}
\phantomsection\label{\detokenize{fspsim.utils:fspsim.utils.Conversions.tle_parse}}
\pysigstartsignatures
\pysiglinewithargsret{\sphinxcode{\sphinxupquote{fspsim.utils.Conversions.}}\sphinxbfcode{\sphinxupquote{tle\_parse}}}{\sphinxparam{\DUrole{n}{tle\_2le}}}{}
\pysigstopsignatures
\sphinxAtStartPar
Parse a 2LE string (e.g. as provided by Celestrak) and return all the data in a dictionary.
\begin{quote}\begin{description}
\sphinxlineitem{Parameters}
\sphinxAtStartPar
\sphinxstyleliteralstrong{\sphinxupquote{tle\_2le}} (\sphinxstyleliteralemphasis{\sphinxupquote{str}}) – 2LE string to be parsed.

\sphinxlineitem{Returns}
\sphinxAtStartPar
Dictionary of all the data contained in the TLE string.

\sphinxlineitem{Return type}
\sphinxAtStartPar
dict

\end{description}\end{quote}

\end{fulllineitems}

\index{true\_to\_eccentric\_anomaly() (in module fspsim.utils.Conversions)@\spxentry{true\_to\_eccentric\_anomaly()}\spxextra{in module fspsim.utils.Conversions}}

\begin{fulllineitems}
\phantomsection\label{\detokenize{fspsim.utils:fspsim.utils.Conversions.true_to_eccentric_anomaly}}
\pysigstartsignatures
\pysiglinewithargsret{\sphinxcode{\sphinxupquote{fspsim.utils.Conversions.}}\sphinxbfcode{\sphinxupquote{true\_to\_eccentric\_anomaly}}}{\sphinxparam{\DUrole{n}{true\_anomaly}}\sphinxparamcomma \sphinxparam{\DUrole{n}{eccentricity}}}{}
\pysigstopsignatures
\sphinxAtStartPar
Convert true anomaly to eccentric anomaly.
\begin{quote}\begin{description}
\sphinxlineitem{Parameters}\begin{itemize}
\item {} 
\sphinxAtStartPar
\sphinxstyleliteralstrong{\sphinxupquote{true\_anomaly}} (\sphinxstyleliteralemphasis{\sphinxupquote{float}}) – True anomaly in radians.

\item {} 
\sphinxAtStartPar
\sphinxstyleliteralstrong{\sphinxupquote{eccentricity}} (\sphinxstyleliteralemphasis{\sphinxupquote{float}}) – Orbital eccentricity.

\end{itemize}

\sphinxlineitem{Returns}
\sphinxAtStartPar
Eccentric anomaly in radians.

\sphinxlineitem{Return type}
\sphinxAtStartPar
float

\end{description}\end{quote}

\end{fulllineitems}

\index{true\_to\_mean\_anomaly() (in module fspsim.utils.Conversions)@\spxentry{true\_to\_mean\_anomaly()}\spxextra{in module fspsim.utils.Conversions}}

\begin{fulllineitems}
\phantomsection\label{\detokenize{fspsim.utils:fspsim.utils.Conversions.true_to_mean_anomaly}}
\pysigstartsignatures
\pysiglinewithargsret{\sphinxcode{\sphinxupquote{fspsim.utils.Conversions.}}\sphinxbfcode{\sphinxupquote{true\_to\_mean\_anomaly}}}{\sphinxparam{\DUrole{n}{true\_anomaly}}\sphinxparamcomma \sphinxparam{\DUrole{n}{eccentricity}}}{}
\pysigstopsignatures
\sphinxAtStartPar
Convert true anomaly to mean anomaly.
\begin{quote}\begin{description}
\sphinxlineitem{Parameters}\begin{itemize}
\item {} 
\sphinxAtStartPar
\sphinxstyleliteralstrong{\sphinxupquote{true\_anomaly}} (\sphinxstyleliteralemphasis{\sphinxupquote{float}}) – True anomaly in radians.

\item {} 
\sphinxAtStartPar
\sphinxstyleliteralstrong{\sphinxupquote{eccentricity}} (\sphinxstyleliteralemphasis{\sphinxupquote{float}}) – Orbital eccentricity.

\end{itemize}

\sphinxlineitem{Returns}
\sphinxAtStartPar
Mean anomaly in radians.

\sphinxlineitem{Return type}
\sphinxAtStartPar
float

\end{description}\end{quote}

\end{fulllineitems}

\index{utc\_to\_jd() (in module fspsim.utils.Conversions)@\spxentry{utc\_to\_jd()}\spxextra{in module fspsim.utils.Conversions}}

\begin{fulllineitems}
\phantomsection\label{\detokenize{fspsim.utils:fspsim.utils.Conversions.utc_to_jd}}
\pysigstartsignatures
\pysiglinewithargsret{\sphinxcode{\sphinxupquote{fspsim.utils.Conversions.}}\sphinxbfcode{\sphinxupquote{utc\_to\_jd}}}{\sphinxparam{\DUrole{n}{time\_stamps}}}{}
\pysigstopsignatures
\sphinxAtStartPar
Convert UTC datetime or string representations to Julian Date (JD).

\sphinxAtStartPar
This function takes in either a single datetime, string representation of a datetime, or a list of them. It then
converts each of these into its corresponding Julian Date (JD) value. If a list is provided, it returns a list of JD
values. If a single datetime or string is provided, it returns a single JD value.
\begin{quote}\begin{description}
\sphinxlineitem{Parameters}
\sphinxAtStartPar
\sphinxstyleliteralstrong{\sphinxupquote{time\_stamps}} (\sphinxstyleliteralemphasis{\sphinxupquote{datetime.datetime}}\sphinxstyleliteralemphasis{\sphinxupquote{, }}\sphinxstyleliteralemphasis{\sphinxupquote{str}}\sphinxstyleliteralemphasis{\sphinxupquote{ or }}\sphinxstyleliteralemphasis{\sphinxupquote{list}}\sphinxstyleliteralemphasis{\sphinxupquote{ of }}\sphinxstyleliteralemphasis{\sphinxupquote{datetime.datetime/str}}) – The datetime object(s) or string representation(s) of dates/times to be converted to Julian Date.
Strings should be in the format ‘\%Y\sphinxhyphen{}\%m\sphinxhyphen{}\%d’ or ‘\%Y\sphinxhyphen{}\%m\sphinxhyphen{}\%d \%H:\%M:\%S’.

\sphinxlineitem{Returns}
\sphinxAtStartPar
The corresponding Julian Date (JD) value(s) for the provided datetime(s) or string representation(s).
Returns a single float if the input is a single datetime or string, and a list of floats if the input is a list.

\sphinxlineitem{Return type}
\sphinxAtStartPar
float or list of float

\end{description}\end{quote}

\end{fulllineitems}

\index{v\_rel() (in module fspsim.utils.Conversions)@\spxentry{v\_rel()}\spxextra{in module fspsim.utils.Conversions}}

\begin{fulllineitems}
\phantomsection\label{\detokenize{fspsim.utils:fspsim.utils.Conversions.v_rel}}
\pysigstartsignatures
\pysiglinewithargsret{\sphinxcode{\sphinxupquote{fspsim.utils.Conversions.}}\sphinxbfcode{\sphinxupquote{v\_rel}}}{\sphinxparam{\DUrole{n}{state}}}{}
\pysigstopsignatures
\sphinxAtStartPar
Calculate the speed of the satellite relative to the Earth’s atmosphere.
This assumes that the atmosphere is stationary and co\sphinxhyphen{}rotating with the surface Earth.
\begin{quote}\begin{description}
\sphinxlineitem{Parameters}
\sphinxAtStartPar
\sphinxstyleliteralstrong{\sphinxupquote{state}} (\sphinxstyleliteralemphasis{\sphinxupquote{array\sphinxhyphen{}like}}) – State vector of the satellite in ECI coordinates.

\sphinxlineitem{Returns}
\sphinxAtStartPar
Velocity of the satellite with respect to the Earth’s atmosphere in Km/s.

\sphinxlineitem{Return type}
\sphinxAtStartPar
np.array

\end{description}\end{quote}

\end{fulllineitems}

\index{write\_tle() (in module fspsim.utils.Conversions)@\spxentry{write\_tle()}\spxextra{in module fspsim.utils.Conversions}}

\begin{fulllineitems}
\phantomsection\label{\detokenize{fspsim.utils:fspsim.utils.Conversions.write_tle}}
\pysigstartsignatures
\pysiglinewithargsret{\sphinxcode{\sphinxupquote{fspsim.utils.Conversions.}}\sphinxbfcode{\sphinxupquote{write\_tle}}}{\sphinxparam{\DUrole{n}{catalog\_number}}\sphinxparamcomma \sphinxparam{\DUrole{n}{classification}}\sphinxparamcomma \sphinxparam{\DUrole{n}{launch\_year}}\sphinxparamcomma \sphinxparam{\DUrole{n}{launch\_number}}\sphinxparamcomma \sphinxparam{\DUrole{n}{launch\_piece}}\sphinxparamcomma \sphinxparam{\DUrole{n}{epoch\_year}}\sphinxparamcomma \sphinxparam{\DUrole{n}{epoch\_day}}\sphinxparamcomma \sphinxparam{\DUrole{n}{first\_derivative}}\sphinxparamcomma \sphinxparam{\DUrole{n}{second\_derivative}}\sphinxparamcomma \sphinxparam{\DUrole{n}{drag\_term}}\sphinxparamcomma \sphinxparam{\DUrole{n}{ephemeris\_type}}\sphinxparamcomma \sphinxparam{\DUrole{n}{element\_set\_number}}\sphinxparamcomma \sphinxparam{\DUrole{n}{inclination}}\sphinxparamcomma \sphinxparam{\DUrole{n}{raan}}\sphinxparamcomma \sphinxparam{\DUrole{n}{eccentricity}}\sphinxparamcomma \sphinxparam{\DUrole{n}{arg\_perigee}}\sphinxparamcomma \sphinxparam{\DUrole{n}{mean\_anomaly}}\sphinxparamcomma \sphinxparam{\DUrole{n}{mean\_motion}}\sphinxparamcomma \sphinxparam{\DUrole{n}{revolution\_number}}}{}
\pysigstopsignatures
\sphinxAtStartPar
This function builds a TLE string in the correct format when given the necessary parameters.
\begin{quote}\begin{description}
\sphinxlineitem{Parameters}\begin{itemize}
\item {} 
\sphinxAtStartPar
\sphinxstyleliteralstrong{\sphinxupquote{catalog\_number}} (\sphinxstyleliteralemphasis{\sphinxupquote{int}}) – Catalog number of the satellite.

\item {} 
\sphinxAtStartPar
\sphinxstyleliteralstrong{\sphinxupquote{classification}} (\sphinxstyleliteralemphasis{\sphinxupquote{str}}) – Classification type (usually ‘U’ for unclassified).

\item {} 
\sphinxAtStartPar
\sphinxstyleliteralstrong{\sphinxupquote{launch\_year}} (\sphinxstyleliteralemphasis{\sphinxupquote{int}}) – Year of launch.

\item {} 
\sphinxAtStartPar
\sphinxstyleliteralstrong{\sphinxupquote{launch\_number}} (\sphinxstyleliteralemphasis{\sphinxupquote{int}}) – Launch number of the year.

\item {} 
\sphinxAtStartPar
\sphinxstyleliteralstrong{\sphinxupquote{launch\_piece}} (\sphinxstyleliteralemphasis{\sphinxupquote{str}}) – Piece of the launch.

\item {} 
\sphinxAtStartPar
\sphinxstyleliteralstrong{\sphinxupquote{epoch\_year}} (\sphinxstyleliteralemphasis{\sphinxupquote{int}}) – Year of the epoch.

\item {} 
\sphinxAtStartPar
\sphinxstyleliteralstrong{\sphinxupquote{epoch\_day}} (\sphinxstyleliteralemphasis{\sphinxupquote{float}}) – Day of year the epoch.

\item {} 
\sphinxAtStartPar
\sphinxstyleliteralstrong{\sphinxupquote{first\_derivative}} (\sphinxstyleliteralemphasis{\sphinxupquote{float}}) – First time derivative of the mean motion (ballistic coefficient).

\item {} 
\sphinxAtStartPar
\sphinxstyleliteralstrong{\sphinxupquote{second\_derivative}} (\sphinxstyleliteralemphasis{\sphinxupquote{float}}) – Second time derivative of the mean motion (delta\sphinxhyphen{}dot).

\item {} 
\sphinxAtStartPar
\sphinxstyleliteralstrong{\sphinxupquote{drag\_term}} (\sphinxstyleliteralemphasis{\sphinxupquote{float}}) – B* drag term.

\item {} 
\sphinxAtStartPar
\sphinxstyleliteralstrong{\sphinxupquote{ephemeris\_type}} (\sphinxstyleliteralemphasis{\sphinxupquote{int}}) – Ephemeris type.

\item {} 
\sphinxAtStartPar
\sphinxstyleliteralstrong{\sphinxupquote{element\_set\_number}} (\sphinxstyleliteralemphasis{\sphinxupquote{int}}) – Element set number.

\item {} 
\sphinxAtStartPar
\sphinxstyleliteralstrong{\sphinxupquote{inclination}} (\sphinxstyleliteralemphasis{\sphinxupquote{float}}) – Inclination (rad).

\item {} 
\sphinxAtStartPar
\sphinxstyleliteralstrong{\sphinxupquote{raan}} (\sphinxstyleliteralemphasis{\sphinxupquote{float}}) – Right ascension of the ascending node (RAAN) (rad).

\item {} 
\sphinxAtStartPar
\sphinxstyleliteralstrong{\sphinxupquote{eccentricity}} (\sphinxstyleliteralemphasis{\sphinxupquote{float}}) – Eccentricity.

\item {} 
\sphinxAtStartPar
\sphinxstyleliteralstrong{\sphinxupquote{arg\_perigee}} (\sphinxstyleliteralemphasis{\sphinxupquote{float}}) – Argument of perigee (rad).

\item {} 
\sphinxAtStartPar
\sphinxstyleliteralstrong{\sphinxupquote{mean\_anomaly}} (\sphinxstyleliteralemphasis{\sphinxupquote{float}}) – Mean anomaly (rad).

\item {} 
\sphinxAtStartPar
\sphinxstyleliteralstrong{\sphinxupquote{mean\_motion}} (\sphinxstyleliteralemphasis{\sphinxupquote{float}}) – Mean motion (rad/s).

\item {} 
\sphinxAtStartPar
\sphinxstyleliteralstrong{\sphinxupquote{revolution\_number}} (\sphinxstyleliteralemphasis{\sphinxupquote{int}}) – Revolution number at epoch.

\end{itemize}

\sphinxlineitem{Returns}
\sphinxAtStartPar
A two\sphinxhyphen{}line element set (TLE) string.

\sphinxlineitem{Return type}
\sphinxAtStartPar
str

\end{description}\end{quote}

\end{fulllineitems}



\paragraph{fspsim.utils.Formatting module}
\label{\detokenize{fspsim.utils:module-fspsim.utils.Formatting}}\label{\detokenize{fspsim.utils:fspsim-utils-formatting-module}}\index{module@\spxentry{module}!fspsim.utils.Formatting@\spxentry{fspsim.utils.Formatting}}\index{fspsim.utils.Formatting@\spxentry{fspsim.utils.Formatting}!module@\spxentry{module}}\index{calculate\_form\_factor() (in module fspsim.utils.Formatting)@\spxentry{calculate\_form\_factor()}\spxextra{in module fspsim.utils.Formatting}}

\begin{fulllineitems}
\phantomsection\label{\detokenize{fspsim.utils:fspsim.utils.Formatting.calculate_form_factor}}
\pysigstartsignatures
\pysiglinewithargsret{\sphinxcode{\sphinxupquote{fspsim.utils.Formatting.}}\sphinxbfcode{\sphinxupquote{calculate\_form\_factor}}}{\sphinxparam{\DUrole{n}{form\_factor\_str}}}{}
\pysigstopsignatures
\sphinxAtStartPar
Convert the form factor specified in the predictions csv to a characteristic length and area.
For CubeSats, the area can be specified by writing the number of “U”s in the form factor string. e.g. “3U” or “6u” (the case does not matter)
For other satellites the form factor must be three numbers (floats or ints), separated by stars. e.g. “1.2*2.3*4.5” or “1*2*3”
\begin{quote}\begin{description}
\sphinxlineitem{Parameters}
\sphinxAtStartPar
\sphinxstyleliteralstrong{\sphinxupquote{form\_factor\_str}} (\sphinxstyleliteralemphasis{\sphinxupquote{string}}) – string describing the form factor of satellties

\sphinxlineitem{Raises}
\sphinxAtStartPar
\sphinxstyleliteralstrong{\sphinxupquote{ValueError}} – Form factor must be a string

\sphinxlineitem{Returns}
\sphinxAtStartPar
characteristic length, characteristic area

\sphinxlineitem{Return type}
\sphinxAtStartPar
tuple

\end{description}\end{quote}

\end{fulllineitems}

\index{future\_constellations\_csv\_handler() (in module fspsim.utils.Formatting)@\spxentry{future\_constellations\_csv\_handler()}\spxextra{in module fspsim.utils.Formatting}}

\begin{fulllineitems}
\phantomsection\label{\detokenize{fspsim.utils:fspsim.utils.Formatting.future_constellations_csv_handler}}
\pysigstartsignatures
\pysiglinewithargsret{\sphinxcode{\sphinxupquote{fspsim.utils.Formatting.}}\sphinxbfcode{\sphinxupquote{future\_constellations\_csv\_handler}}}{\sphinxparam{\DUrole{n}{file\_path}}}{}
\pysigstopsignatures
\sphinxAtStartPar
Checks that the user supplied Future Constellation CSV is in the correct format for the simulation
\begin{quote}\begin{description}
\sphinxlineitem{Parameters}
\sphinxAtStartPar
\sphinxstyleliteralstrong{\sphinxupquote{file\_path}} (\sphinxstyleliteralemphasis{\sphinxupquote{str}}) – File Path of the CSV

\sphinxlineitem{Returns}
\sphinxAtStartPar
Dictionary of the constellations in a format the fspsim can read.

\sphinxlineitem{Return type}
\sphinxAtStartPar
dict

\end{description}\end{quote}

\end{fulllineitems}



\paragraph{fspsim.utils.LaunchModel module}
\label{\detokenize{fspsim.utils:module-fspsim.utils.LaunchModel}}\label{\detokenize{fspsim.utils:fspsim-utils-launchmodel-module}}\index{module@\spxentry{module}!fspsim.utils.LaunchModel@\spxentry{fspsim.utils.LaunchModel}}\index{fspsim.utils.LaunchModel@\spxentry{fspsim.utils.LaunchModel}!module@\spxentry{module}}\index{Prediction2SpaceObjects() (in module fspsim.utils.LaunchModel)@\spxentry{Prediction2SpaceObjects()}\spxextra{in module fspsim.utils.LaunchModel}}

\begin{fulllineitems}
\phantomsection\label{\detokenize{fspsim.utils:fspsim.utils.LaunchModel.Prediction2SpaceObjects}}
\pysigstartsignatures
\pysiglinewithargsret{\sphinxcode{\sphinxupquote{fspsim.utils.LaunchModel.}}\sphinxbfcode{\sphinxupquote{Prediction2SpaceObjects}}}{\sphinxparam{\DUrole{n}{satellite\_predictions\_csv}}\sphinxparamcomma \sphinxparam{\DUrole{n}{simsettings}}}{}
\pysigstopsignatures
\sphinxAtStartPar
Generates a list SpaceObjects for each of the sub\sphinxhyphen{}constellation in the prediction data CSV file.
Each object gets assigned a launch and decay date based on the launch model parameters and the parameters
specified in the prediction data CSV file.
\begin{quote}\begin{description}
\sphinxlineitem{Parameters}\begin{itemize}
\item {} 
\sphinxAtStartPar
\sphinxstyleliteralstrong{\sphinxupquote{satellite\_predictions\_csv}} (\sphinxstyleliteralemphasis{\sphinxupquote{str}}) – CSV file with the satellite predictions data.

\item {} 
\sphinxAtStartPar
\sphinxstyleliteralstrong{\sphinxupquote{simsettings}} (\sphinxstyleliteralemphasis{\sphinxupquote{dict}}) – Loaded JSON file with the simulation settings (contains launch model parameters).

\end{itemize}

\sphinxlineitem{Returns}
\sphinxAtStartPar
A list of space objects generated from the prediction data.

\sphinxlineitem{Return type}
\sphinxAtStartPar
list{[}{\hyperref[\detokenize{fspsim.utils:fspsim.utils.SpaceObject.SpaceObject}]{\sphinxcrossref{SpaceObject}}}{]}

\end{description}\end{quote}

\end{fulllineitems}

\index{create\_subconstellation\_Space\_Objects() (in module fspsim.utils.LaunchModel)@\spxentry{create\_subconstellation\_Space\_Objects()}\spxextra{in module fspsim.utils.LaunchModel}}

\begin{fulllineitems}
\phantomsection\label{\detokenize{fspsim.utils:fspsim.utils.LaunchModel.create_subconstellation_Space_Objects}}
\pysigstartsignatures
\pysiglinewithargsret{\sphinxcode{\sphinxupquote{fspsim.utils.LaunchModel.}}\sphinxbfcode{\sphinxupquote{create\_subconstellation\_Space\_Objects}}}{\sphinxparam{\DUrole{n}{N}}\sphinxparamcomma \sphinxparam{\DUrole{n}{i}}\sphinxparamcomma \sphinxparam{\DUrole{n}{h}}\sphinxparamcomma \sphinxparam{\DUrole{n}{\_soname}}\sphinxparamcomma \sphinxparam{\DUrole{n}{\_application}}\sphinxparamcomma \sphinxparam{\DUrole{n}{\_owner}}\sphinxparamcomma \sphinxparam{\DUrole{n}{launch\_schedule}}\sphinxparamcomma \sphinxparam{\DUrole{n}{\_mass}}\sphinxparamcomma \sphinxparam{\DUrole{n}{\_area}}\sphinxparamcomma \sphinxparam{\DUrole{n}{\_length}}\sphinxparamcomma \sphinxparam{\DUrole{n}{\_maneuverable}}\sphinxparamcomma \sphinxparam{\DUrole{n}{\_propulsion}}}{}
\pysigstopsignatures
\sphinxAtStartPar
Generate a list of SpaceObjects that are in a Walker\sphinxhyphen{}Delta constellation.
This maximizes geometric coverage of the Earth given the number of satellites and the inclination and altitude of the orbit.
Based on the parameters passed in the launch file.
\begin{quote}\begin{description}
\sphinxlineitem{Parameters}\begin{itemize}
\item {} 
\sphinxAtStartPar
\sphinxstyleliteralstrong{\sphinxupquote{N}} (\sphinxstyleliteralemphasis{\sphinxupquote{int}}) – Number of satellites in the constellation.

\item {} 
\sphinxAtStartPar
\sphinxstyleliteralstrong{\sphinxupquote{i}} (\sphinxstyleliteralemphasis{\sphinxupquote{float}}) – Inclination of the satellite orbit.

\item {} 
\sphinxAtStartPar
\sphinxstyleliteralstrong{\sphinxupquote{h}} (\sphinxstyleliteralemphasis{\sphinxupquote{float}}) – Altitude of the satellite orbit.

\item {} 
\sphinxAtStartPar
\sphinxstyleliteralstrong{\sphinxupquote{\_soname}} (\sphinxstyleliteralemphasis{\sphinxupquote{str}}) – Name of the sub\sphinxhyphen{}constellation.

\item {} 
\sphinxAtStartPar
\sphinxstyleliteralstrong{\sphinxupquote{\_application}} (\sphinxstyleliteralemphasis{\sphinxupquote{str}}) – Application for which the satellite is used.

\item {} 
\sphinxAtStartPar
\sphinxstyleliteralstrong{\sphinxupquote{\_owner}} (\sphinxstyleliteralemphasis{\sphinxupquote{str}}) – Owner or operator of the satellite.

\item {} 
\sphinxAtStartPar
\sphinxstyleliteralstrong{\sphinxupquote{launch\_schedule}} (\sphinxstyleliteralemphasis{\sphinxupquote{list}}\sphinxstyleliteralemphasis{\sphinxupquote{{[}}}\sphinxstyleliteralemphasis{\sphinxupquote{str}}\sphinxstyleliteralemphasis{\sphinxupquote{{]}}}) – Schedule of satellite launches.

\item {} 
\sphinxAtStartPar
\sphinxstyleliteralstrong{\sphinxupquote{\_mass}} (\sphinxstyleliteralemphasis{\sphinxupquote{float}}) – Mass of the satellite.

\item {} 
\sphinxAtStartPar
\sphinxstyleliteralstrong{\sphinxupquote{\_area}} (\sphinxstyleliteralemphasis{\sphinxupquote{float}}) – Characteristic area of the satellite.

\item {} 
\sphinxAtStartPar
\sphinxstyleliteralstrong{\sphinxupquote{\_length}} (\sphinxstyleliteralemphasis{\sphinxupquote{float}}) – Characteristic length of the satellite.

\item {} 
\sphinxAtStartPar
\sphinxstyleliteralstrong{\sphinxupquote{\_maneuverable}} (\sphinxstyleliteralemphasis{\sphinxupquote{str}}) – Indicates if the satellite is maneuverable.

\item {} 
\sphinxAtStartPar
\sphinxstyleliteralstrong{\sphinxupquote{\_propulsion}} (\sphinxstyleliteralemphasis{\sphinxupquote{str}}) – Type of propulsion used in the satellite.

\end{itemize}

\sphinxlineitem{Raises}
\sphinxAtStartPar
\sphinxstyleliteralstrong{\sphinxupquote{ValueError}} – If N is less than 1.

\sphinxlineitem{Returns}
\sphinxAtStartPar
A list of space objects for the sub\sphinxhyphen{}constellation.

\sphinxlineitem{Return type}
\sphinxAtStartPar
list{[}{\hyperref[\detokenize{fspsim.utils:fspsim.utils.SpaceObject.SpaceObject}]{\sphinxcrossref{SpaceObject}}}{]}

\end{description}\end{quote}

\end{fulllineitems}

\index{global\_launch\_schedule() (in module fspsim.utils.LaunchModel)@\spxentry{global\_launch\_schedule()}\spxextra{in module fspsim.utils.LaunchModel}}

\begin{fulllineitems}
\phantomsection\label{\detokenize{fspsim.utils:fspsim.utils.LaunchModel.global_launch_schedule}}
\pysigstartsignatures
\pysiglinewithargsret{\sphinxcode{\sphinxupquote{fspsim.utils.LaunchModel.}}\sphinxbfcode{\sphinxupquote{global\_launch\_schedule}}}{\sphinxparam{\DUrole{n}{sub\_constellation\_metadata\_dicts}}\sphinxparamcomma \sphinxparam{\DUrole{n}{monthly\_ton\_capacity}}\sphinxparamcomma \sphinxparam{\DUrole{n}{launches\_start\_date}}\sphinxparamcomma \sphinxparam{\DUrole{n}{rocket}\DUrole{o}{=}\DUrole{default_value}{'Falcon 9'}}}{}
\pysigstopsignatures
\sphinxAtStartPar
Determines the launch dates for various sub\sphinxhyphen{}constellations based on a greedy algorithm, limited by
global montly launch capacity (tons/month), the date to start scheduling launches, and the type of rocket used.
\begin{quote}\begin{description}
\sphinxlineitem{Parameters}\begin{itemize}
\item {} 
\sphinxAtStartPar
\sphinxstyleliteralstrong{\sphinxupquote{sub\_constellation\_metadata\_dicts}} (\sphinxstyleliteralemphasis{\sphinxupquote{list}}\sphinxstyleliteralemphasis{\sphinxupquote{{[}}}\sphinxstyleliteralemphasis{\sphinxupquote{dict}}\sphinxstyleliteralemphasis{\sphinxupquote{{]}}}) – List of metadata for each sub\sphinxhyphen{}constellation.

\item {} 
\sphinxAtStartPar
\sphinxstyleliteralstrong{\sphinxupquote{monthly\_ton\_capacity}} (\sphinxstyleliteralemphasis{\sphinxupquote{float}}) – The maximum weight capacity available for launching in tons per month.

\item {} 
\sphinxAtStartPar
\sphinxstyleliteralstrong{\sphinxupquote{launches\_start\_date}} (\sphinxstyleliteralemphasis{\sphinxupquote{str}}) – The initial date to begin scheduling launches.

\item {} 
\sphinxAtStartPar
\sphinxstyleliteralstrong{\sphinxupquote{rocket}} (\sphinxstyleliteralemphasis{\sphinxupquote{str}}\sphinxstyleliteralemphasis{\sphinxupquote{, }}\sphinxstyleliteralemphasis{\sphinxupquote{optional}}) – The type of rocket used for launches, defaults to “Falcon 9”.

\end{itemize}

\sphinxlineitem{Raises}
\sphinxAtStartPar
\sphinxstyleliteralstrong{\sphinxupquote{ValueError}} – If the provided rocket is not in the list of LEO launchers.

\sphinxlineitem{Returns}
\sphinxAtStartPar
A dictionary of launch dates for each sub\sphinxhyphen{}constellation.

\sphinxlineitem{Return type}
\sphinxAtStartPar
dict{[}str, list{[}str{]}{]}

\end{description}\end{quote}

\end{fulllineitems}

\index{import\_configuration\_json() (in module fspsim.utils.LaunchModel)@\spxentry{import\_configuration\_json()}\spxextra{in module fspsim.utils.LaunchModel}}

\begin{fulllineitems}
\phantomsection\label{\detokenize{fspsim.utils:fspsim.utils.LaunchModel.import_configuration_json}}
\pysigstartsignatures
\pysiglinewithargsret{\sphinxcode{\sphinxupquote{fspsim.utils.LaunchModel.}}\sphinxbfcode{\sphinxupquote{import\_configuration\_json}}}{\sphinxparam{\DUrole{n}{filename}}}{}
\pysigstopsignatures
\end{fulllineitems}



\paragraph{fspsim.utils.Propagators module}
\label{\detokenize{fspsim.utils:module-fspsim.utils.Propagators}}\label{\detokenize{fspsim.utils:fspsim-utils-propagators-module}}\index{module@\spxentry{module}!fspsim.utils.Propagators@\spxentry{fspsim.utils.Propagators}}\index{fspsim.utils.Propagators@\spxentry{fspsim.utils.Propagators}!module@\spxentry{module}}\index{compute\_eccentric\_anomaly() (in module fspsim.utils.Propagators)@\spxentry{compute\_eccentric\_anomaly()}\spxextra{in module fspsim.utils.Propagators}}

\begin{fulllineitems}
\phantomsection\label{\detokenize{fspsim.utils:fspsim.utils.Propagators.compute_eccentric_anomaly}}
\pysigstartsignatures
\pysiglinewithargsret{\sphinxcode{\sphinxupquote{fspsim.utils.Propagators.}}\sphinxbfcode{\sphinxupquote{compute\_eccentric\_anomaly}}}{\sphinxparam{\DUrole{n}{Mo}}\sphinxparamcomma \sphinxparam{\DUrole{n}{e}}\sphinxparamcomma \sphinxparam{\DUrole{n}{n}}\sphinxparamcomma \sphinxparam{\DUrole{n}{step}}\sphinxparamcomma \sphinxparam{\DUrole{n}{step\_size}}\sphinxparamcomma \sphinxparam{\DUrole{n}{r\_tol}}}{}
\pysigstopsignatures
\sphinxAtStartPar
Computes the eccentric anomaly using Newton’s method.
\begin{quote}\begin{description}
\sphinxlineitem{Parameters}\begin{itemize}
\item {} 
\sphinxAtStartPar
\sphinxstyleliteralstrong{\sphinxupquote{Mo}} (\sphinxstyleliteralemphasis{\sphinxupquote{float}}) – Mean anomaly (first “guess”) in radians

\item {} 
\sphinxAtStartPar
\sphinxstyleliteralstrong{\sphinxupquote{e}} (\sphinxstyleliteralemphasis{\sphinxupquote{float}}) – Eccentricity of the orbit

\item {} 
\sphinxAtStartPar
\sphinxstyleliteralstrong{\sphinxupquote{n}} (\sphinxstyleliteralemphasis{\sphinxupquote{float}}) – Mean motion of the orbit in radians/second

\item {} 
\sphinxAtStartPar
\sphinxstyleliteralstrong{\sphinxupquote{step}} (\sphinxstyleliteralemphasis{\sphinxupquote{int}}) – Current step number

\item {} 
\sphinxAtStartPar
\sphinxstyleliteralstrong{\sphinxupquote{step\_size}} (\sphinxstyleliteralemphasis{\sphinxupquote{float}}) – Time step of propagation in seconds

\item {} 
\sphinxAtStartPar
\sphinxstyleliteralstrong{\sphinxupquote{r\_tol}} (\sphinxstyleliteralemphasis{\sphinxupquote{float}}) – Relative tolerance for Newton’s method

\end{itemize}

\sphinxlineitem{Returns}
\sphinxAtStartPar
\sphinxstylestrong{E} – Approximated value of eccentric anomaly in radians

\sphinxlineitem{Return type}
\sphinxAtStartPar
float

\end{description}\end{quote}

\end{fulllineitems}

\index{compute\_gaussian\_vectors() (in module fspsim.utils.Propagators)@\spxentry{compute\_gaussian\_vectors()}\spxextra{in module fspsim.utils.Propagators}}

\begin{fulllineitems}
\phantomsection\label{\detokenize{fspsim.utils:fspsim.utils.Propagators.compute_gaussian_vectors}}
\pysigstartsignatures
\pysiglinewithargsret{\sphinxcode{\sphinxupquote{fspsim.utils.Propagators.}}\sphinxbfcode{\sphinxupquote{compute\_gaussian\_vectors}}}{\sphinxparam{\DUrole{n}{W}}\sphinxparamcomma \sphinxparam{\DUrole{n}{w}}\sphinxparamcomma \sphinxparam{\DUrole{n}{i}}}{}
\pysigstopsignatures
\sphinxAtStartPar
Computes the Gaussian vectors P and Q for coordinate transformation.
\begin{quote}\begin{description}
\sphinxlineitem{Parameters}\begin{itemize}
\item {} 
\sphinxAtStartPar
\sphinxstyleliteralstrong{\sphinxupquote{W}} (\sphinxstyleliteralemphasis{\sphinxupquote{float}}) – Right ascension of the ascending node of the orbit in radians

\item {} 
\sphinxAtStartPar
\sphinxstyleliteralstrong{\sphinxupquote{w}} (\sphinxstyleliteralemphasis{\sphinxupquote{float}}) – Argument of perigee of the orbit in radians

\item {} 
\sphinxAtStartPar
\sphinxstyleliteralstrong{\sphinxupquote{i}} (\sphinxstyleliteralemphasis{\sphinxupquote{float}}) – Inclination of the orbit in radians

\end{itemize}

\sphinxlineitem{Returns}
\sphinxAtStartPar
\begin{itemize}
\item {} 
\sphinxAtStartPar
\sphinxstylestrong{P} (\sphinxstyleemphasis{numpy.ndarray}) – Gaussian vector P

\item {} 
\sphinxAtStartPar
\sphinxstylestrong{Q} (\sphinxstyleemphasis{numpy.ndarray}) – Gaussian vector Q

\end{itemize}


\end{description}\end{quote}

\end{fulllineitems}

\index{compute\_velocity() (in module fspsim.utils.Propagators)@\spxentry{compute\_velocity()}\spxextra{in module fspsim.utils.Propagators}}

\begin{fulllineitems}
\phantomsection\label{\detokenize{fspsim.utils:fspsim.utils.Propagators.compute_velocity}}
\pysigstartsignatures
\pysiglinewithargsret{\sphinxcode{\sphinxupquote{fspsim.utils.Propagators.}}\sphinxbfcode{\sphinxupquote{compute\_velocity}}}{\sphinxparam{\DUrole{n}{GM\_earth}}\sphinxparamcomma \sphinxparam{\DUrole{n}{a}}\sphinxparamcomma \sphinxparam{\DUrole{n}{e}}\sphinxparamcomma \sphinxparam{\DUrole{n}{E}}\sphinxparamcomma \sphinxparam{\DUrole{n}{P}}\sphinxparamcomma \sphinxparam{\DUrole{n}{Q}}}{}
\pysigstopsignatures
\sphinxAtStartPar
Computes the velocity vector of a satellite in a Keplerian orbit.
\begin{quote}\begin{description}
\sphinxlineitem{Parameters}\begin{itemize}
\item {} 
\sphinxAtStartPar
\sphinxstyleliteralstrong{\sphinxupquote{GM\_earth}} (\sphinxstyleliteralemphasis{\sphinxupquote{float}}) – Gravitational constant of the Earth in km\textasciicircum{}3/s\textasciicircum{}2

\item {} 
\sphinxAtStartPar
\sphinxstyleliteralstrong{\sphinxupquote{a}} (\sphinxstyleliteralemphasis{\sphinxupquote{float}}) – Semi\sphinxhyphen{}major axis of the orbit in km

\item {} 
\sphinxAtStartPar
\sphinxstyleliteralstrong{\sphinxupquote{e}} (\sphinxstyleliteralemphasis{\sphinxupquote{float}}) – Eccentricity of the orbit

\item {} 
\sphinxAtStartPar
\sphinxstyleliteralstrong{\sphinxupquote{E}} (\sphinxstyleliteralemphasis{\sphinxupquote{float}}) – Eccentric anomaly in radians

\item {} 
\sphinxAtStartPar
\sphinxstyleliteralstrong{\sphinxupquote{P}} (\sphinxstyleliteralemphasis{\sphinxupquote{numpy.ndarray}}) – Gaussian vector P

\item {} 
\sphinxAtStartPar
\sphinxstyleliteralstrong{\sphinxupquote{Q}} (\sphinxstyleliteralemphasis{\sphinxupquote{numpy.ndarray}}) – Gaussian vector Q

\end{itemize}

\sphinxlineitem{Returns}
\sphinxAtStartPar
Velocity vector of the satellite in km/s

\sphinxlineitem{Return type}
\sphinxAtStartPar
numpy.ndarray

\end{description}\end{quote}

\end{fulllineitems}

\index{kepler\_prop() (in module fspsim.utils.Propagators)@\spxentry{kepler\_prop()}\spxextra{in module fspsim.utils.Propagators}}

\begin{fulllineitems}
\phantomsection\label{\detokenize{fspsim.utils:fspsim.utils.Propagators.kepler_prop}}
\pysigstartsignatures
\pysiglinewithargsret{\sphinxcode{\sphinxupquote{fspsim.utils.Propagators.}}\sphinxbfcode{\sphinxupquote{kepler\_prop}}}{\sphinxparam{\DUrole{n}{jd\_start}}\sphinxparamcomma \sphinxparam{\DUrole{n}{jd\_stop}}\sphinxparamcomma \sphinxparam{\DUrole{n}{step\_size}}\sphinxparamcomma \sphinxparam{\DUrole{n}{a}}\sphinxparamcomma \sphinxparam{\DUrole{n}{e}}\sphinxparamcomma \sphinxparam{\DUrole{n}{i}}\sphinxparamcomma \sphinxparam{\DUrole{n}{w}}\sphinxparamcomma \sphinxparam{\DUrole{n}{W}}\sphinxparamcomma \sphinxparam{\DUrole{n}{V}}\sphinxparamcomma \sphinxparam{\DUrole{n}{area}\DUrole{o}{=}\DUrole{default_value}{None}}\sphinxparamcomma \sphinxparam{\DUrole{n}{mass}\DUrole{o}{=}\DUrole{default_value}{None}}\sphinxparamcomma \sphinxparam{\DUrole{n}{cd}\DUrole{o}{=}\DUrole{default_value}{None}}\sphinxparamcomma \sphinxparam{\DUrole{n}{drag\_decay}\DUrole{o}{=}\DUrole{default_value}{False}}}{}
\pysigstopsignatures
\sphinxAtStartPar
Propagates the orbit of a satellite in a Keplerian orbit, taking drag decay into account.
Uses Kepler’s equation for propagation and Gaussian vectors for coordinate transformation.
Stops the propagation if altitude drops below 200 km.
Area, mass, and drag coefficient are required only for drag decay.
\begin{quote}\begin{description}
\sphinxlineitem{Parameters}\begin{itemize}
\item {} 
\sphinxAtStartPar
\sphinxstyleliteralstrong{\sphinxupquote{jd\_start}} (\sphinxstyleliteralemphasis{\sphinxupquote{float}}) – start time of propagation in Julian Date format

\item {} 
\sphinxAtStartPar
\sphinxstyleliteralstrong{\sphinxupquote{jd\_stop}} (\sphinxstyleliteralemphasis{\sphinxupquote{float}}) – end time of propagation in Julian Date format

\item {} 
\sphinxAtStartPar
\sphinxstyleliteralstrong{\sphinxupquote{step\_size}} (\sphinxstyleliteralemphasis{\sphinxupquote{float}}) – time step of propagation in seconds

\item {} 
\sphinxAtStartPar
\sphinxstyleliteralstrong{\sphinxupquote{a}} (\sphinxstyleliteralemphasis{\sphinxupquote{float}}) – semi\sphinxhyphen{}major axis of the orbit in km

\item {} 
\sphinxAtStartPar
\sphinxstyleliteralstrong{\sphinxupquote{e}} (\sphinxstyleliteralemphasis{\sphinxupquote{float}}) – eccentricity of the orbit

\item {} 
\sphinxAtStartPar
\sphinxstyleliteralstrong{\sphinxupquote{i}} (\sphinxstyleliteralemphasis{\sphinxupquote{float}}) – inclination of the orbit in degrees

\item {} 
\sphinxAtStartPar
\sphinxstyleliteralstrong{\sphinxupquote{w}} (\sphinxstyleliteralemphasis{\sphinxupquote{float}}) – argument of perigee of the orbit in degrees

\item {} 
\sphinxAtStartPar
\sphinxstyleliteralstrong{\sphinxupquote{W}} (\sphinxstyleliteralemphasis{\sphinxupquote{float}}) – right ascension of the ascending node of the orbit in degrees

\item {} 
\sphinxAtStartPar
\sphinxstyleliteralstrong{\sphinxupquote{V}} (\sphinxstyleliteralemphasis{\sphinxupquote{float}}) – true anomaly of the orbit in degrees

\item {} 
\sphinxAtStartPar
\sphinxstyleliteralstrong{\sphinxupquote{area}} (\sphinxstyleliteralemphasis{\sphinxupquote{float}}\sphinxstyleliteralemphasis{\sphinxupquote{, }}\sphinxstyleliteralemphasis{\sphinxupquote{optional}}) – cross\sphinxhyphen{}sectional area of the satellite in m\textasciicircum{}2. Defaults to None.

\item {} 
\sphinxAtStartPar
\sphinxstyleliteralstrong{\sphinxupquote{mass}} (\sphinxstyleliteralemphasis{\sphinxupquote{float}}\sphinxstyleliteralemphasis{\sphinxupquote{, }}\sphinxstyleliteralemphasis{\sphinxupquote{optional}}) – mass of the satellite in kg. Defaults to None.

\item {} 
\sphinxAtStartPar
\sphinxstyleliteralstrong{\sphinxupquote{cd}} (\sphinxstyleliteralemphasis{\sphinxupquote{float}}\sphinxstyleliteralemphasis{\sphinxupquote{, }}\sphinxstyleliteralemphasis{\sphinxupquote{optional}}) – drag coefficient of the satellite. Defaults to None.

\item {} 
\sphinxAtStartPar
\sphinxstyleliteralstrong{\sphinxupquote{drag\_decay}} (\sphinxstyleliteralemphasis{\sphinxupquote{bool}}\sphinxstyleliteralemphasis{\sphinxupquote{, }}\sphinxstyleliteralemphasis{\sphinxupquote{optional}}) – If True, decay the semi\sphinxhyphen{}major axis due to drag. Defaults to False.

\end{itemize}

\end{description}\end{quote}

\sphinxAtStartPar
Returns:
list: list of lists containing the propagated ephemeris

\end{fulllineitems}

\index{sgp4\_prop\_TLE() (in module fspsim.utils.Propagators)@\spxentry{sgp4\_prop\_TLE()}\spxextra{in module fspsim.utils.Propagators}}

\begin{fulllineitems}
\phantomsection\label{\detokenize{fspsim.utils:fspsim.utils.Propagators.sgp4_prop_TLE}}
\pysigstartsignatures
\pysiglinewithargsret{\sphinxcode{\sphinxupquote{fspsim.utils.Propagators.}}\sphinxbfcode{\sphinxupquote{sgp4\_prop\_TLE}}}{\sphinxparam{\DUrole{n}{TLE}}\sphinxparamcomma \sphinxparam{\DUrole{n}{jd\_start}}\sphinxparamcomma \sphinxparam{\DUrole{n}{jd\_end}}\sphinxparamcomma \sphinxparam{\DUrole{n}{dt}}}{}
\pysigstopsignatures
\sphinxAtStartPar
Given a TLE, a start time, end time, and time step, propagate the TLE and return the time\sphinxhyphen{}series of Cartesian coordinates, and accompanying time\sphinxhyphen{}stamps (MJD)
Note: Simply a wrapper for the SGP4 routine in the sgp4.api package (Brandon Rhodes)
\begin{quote}\begin{description}
\sphinxlineitem{Parameters}\begin{itemize}
\item {} 
\sphinxAtStartPar
\sphinxstyleliteralstrong{\sphinxupquote{TLE}} (\sphinxstyleliteralemphasis{\sphinxupquote{string}}) – TLE to be propagated

\item {} 
\sphinxAtStartPar
\sphinxstyleliteralstrong{\sphinxupquote{jd\_start}} (\sphinxstyleliteralemphasis{\sphinxupquote{float}}) – start time of propagation in Julian Date format

\item {} 
\sphinxAtStartPar
\sphinxstyleliteralstrong{\sphinxupquote{jd\_end}} (\sphinxstyleliteralemphasis{\sphinxupquote{float}}) – end time of propagation in Julian Date format

\item {} 
\sphinxAtStartPar
\sphinxstyleliteralstrong{\sphinxupquote{dt}} (\sphinxstyleliteralemphasis{\sphinxupquote{float}}) – time step of propagation in seconds

\item {} 
\sphinxAtStartPar
\sphinxstyleliteralstrong{\sphinxupquote{alt\_series}} (\sphinxstyleliteralemphasis{\sphinxupquote{bool}}\sphinxstyleliteralemphasis{\sphinxupquote{, }}\sphinxstyleliteralemphasis{\sphinxupquote{optional}}) – If True, return the altitude series as well as the position series. Defaults to False.

\end{itemize}

\end{description}\end{quote}

\sphinxAtStartPar
Returns:
list: list of lists containing the time\sphinxhyphen{}series of Cartesian coordinates, and accompanying time\sphinxhyphen{}stamps (JD)

\end{fulllineitems}



\paragraph{fspsim.utils.SpaceCatalogue module}
\label{\detokenize{fspsim.utils:module-fspsim.utils.SpaceCatalogue}}\label{\detokenize{fspsim.utils:fspsim-utils-spacecatalogue-module}}\index{module@\spxentry{module}!fspsim.utils.SpaceCatalogue@\spxentry{fspsim.utils.SpaceCatalogue}}\index{fspsim.utils.SpaceCatalogue@\spxentry{fspsim.utils.SpaceCatalogue}!module@\spxentry{module}}\index{SpaceCatalogue (class in fspsim.utils.SpaceCatalogue)@\spxentry{SpaceCatalogue}\spxextra{class in fspsim.utils.SpaceCatalogue}}

\begin{fulllineitems}
\phantomsection\label{\detokenize{fspsim.utils:fspsim.utils.SpaceCatalogue.SpaceCatalogue}}
\pysigstartsignatures
\pysiglinewithargsret{\sphinxbfcode{\sphinxupquote{class\DUrole{w}{ }}}\sphinxcode{\sphinxupquote{fspsim.utils.SpaceCatalogue.}}\sphinxbfcode{\sphinxupquote{SpaceCatalogue}}}{\sphinxparam{\DUrole{n}{settings}}\sphinxparamcomma \sphinxparam{\DUrole{n}{future\_constellations}\DUrole{o}{=}\DUrole{default_value}{None}}}{}
\pysigstopsignatures
\sphinxAtStartPar
Bases: \sphinxcode{\sphinxupquote{object}}

\sphinxAtStartPar
A representation of a space catalogue that consolidates satellite data from various sources.

\sphinxAtStartPar
The class aims to process and manage satellite information from multiple sources like JSR and SpaceTrack.
It provides functionality to merge active satellite catalogues, repull information from sources,
and create a consolidated catalogue of space objects.
\index{sim\_object\_type (fspsim.utils.SpaceCatalogue.SpaceCatalogue attribute)@\spxentry{sim\_object\_type}\spxextra{fspsim.utils.SpaceCatalogue.SpaceCatalogue attribute}}

\begin{fulllineitems}
\phantomsection\label{\detokenize{fspsim.utils:fspsim.utils.SpaceCatalogue.SpaceCatalogue.sim_object_type}}
\pysigstartsignatures
\pysigline{\sphinxbfcode{\sphinxupquote{sim\_object\_type}}}
\pysigstopsignatures
\sphinxAtStartPar
Type of simulation objects. Can be ‘active’, ‘all’, or ‘debris’.
\begin{quote}\begin{description}
\sphinxlineitem{Type}
\sphinxAtStartPar
str

\end{description}\end{quote}

\end{fulllineitems}

\index{sim\_object\_catalogue (fspsim.utils.SpaceCatalogue.SpaceCatalogue attribute)@\spxentry{sim\_object\_catalogue}\spxextra{fspsim.utils.SpaceCatalogue.SpaceCatalogue attribute}}

\begin{fulllineitems}
\phantomsection\label{\detokenize{fspsim.utils:fspsim.utils.SpaceCatalogue.SpaceCatalogue.sim_object_catalogue}}
\pysigstartsignatures
\pysigline{\sphinxbfcode{\sphinxupquote{sim\_object\_catalogue}}}
\pysigstopsignatures
\sphinxAtStartPar
Source of the simulation object. Can be ‘jsr’, ‘spacetrack’, or ‘both’.
\begin{quote}\begin{description}
\sphinxlineitem{Type}
\sphinxAtStartPar
str

\end{description}\end{quote}

\end{fulllineitems}

\index{Catalogue (fspsim.utils.SpaceCatalogue.SpaceCatalogue attribute)@\spxentry{Catalogue}\spxextra{fspsim.utils.SpaceCatalogue.SpaceCatalogue attribute}}

\begin{fulllineitems}
\phantomsection\label{\detokenize{fspsim.utils:fspsim.utils.SpaceCatalogue.SpaceCatalogue.Catalogue}}
\pysigstartsignatures
\pysigline{\sphinxbfcode{\sphinxupquote{Catalogue}}}
\pysigstopsignatures
\sphinxAtStartPar
A list of SpaceObjects representing the space objects in the catalogue.
\begin{quote}\begin{description}
\sphinxlineitem{Type}
\sphinxAtStartPar
list

\end{description}\end{quote}

\end{fulllineitems}

\index{repull\_catalogues (fspsim.utils.SpaceCatalogue.SpaceCatalogue attribute)@\spxentry{repull\_catalogues}\spxextra{fspsim.utils.SpaceCatalogue.SpaceCatalogue attribute}}

\begin{fulllineitems}
\phantomsection\label{\detokenize{fspsim.utils:fspsim.utils.SpaceCatalogue.SpaceCatalogue.repull_catalogues}}
\pysigstartsignatures
\pysigline{\sphinxbfcode{\sphinxupquote{repull\_catalogues}}}
\pysigstopsignatures
\sphinxAtStartPar
Flag to determine if the catalogues should be repulled from sources.
\begin{quote}\begin{description}
\sphinxlineitem{Type}
\sphinxAtStartPar
bool

\end{description}\end{quote}

\end{fulllineitems}

\begin{quote}\begin{description}
\sphinxlineitem{Raises}\begin{itemize}
\item {} 
\sphinxAtStartPar
\sphinxstyleliteralstrong{\sphinxupquote{Exception}} – If an invalid sim\_object\_type or sim\_object\_catalogue is specified.

\item {} 
\sphinxAtStartPar
\sphinxstyleliteralstrong{\sphinxupquote{TypeError}} – If loaded data from file is not an instance of SpaceCatalogue.

\end{itemize}

\end{description}\end{quote}
\index{Catalogue2SpaceObjects() (fspsim.utils.SpaceCatalogue.SpaceCatalogue method)@\spxentry{Catalogue2SpaceObjects()}\spxextra{fspsim.utils.SpaceCatalogue.SpaceCatalogue method}}

\begin{fulllineitems}
\phantomsection\label{\detokenize{fspsim.utils:fspsim.utils.SpaceCatalogue.SpaceCatalogue.Catalogue2SpaceObjects}}
\pysigstartsignatures
\pysiglinewithargsret{\sphinxbfcode{\sphinxupquote{Catalogue2SpaceObjects}}}{}{}
\pysigstopsignatures
\sphinxAtStartPar
Convert the current catalogue into a list of SpaceObjects.
\begin{quote}\begin{description}
\sphinxlineitem{Raises}
\sphinxAtStartPar
\sphinxstyleliteralstrong{\sphinxupquote{Exception}} – If invalid \sphinxtitleref{sim\_object\_type} is specified.

\sphinxlineitem{Returns}
\sphinxAtStartPar
List of SpaceObjects.

\sphinxlineitem{Return type}
\sphinxAtStartPar
list{[}{\hyperref[\detokenize{fspsim.utils:fspsim.utils.SpaceObject.SpaceObject}]{\sphinxcrossref{SpaceObject}}}{]}

\end{description}\end{quote}

\end{fulllineitems}

\index{CreateCatalogueActive() (fspsim.utils.SpaceCatalogue.SpaceCatalogue method)@\spxentry{CreateCatalogueActive()}\spxextra{fspsim.utils.SpaceCatalogue.SpaceCatalogue method}}

\begin{fulllineitems}
\phantomsection\label{\detokenize{fspsim.utils:fspsim.utils.SpaceCatalogue.SpaceCatalogue.CreateCatalogueActive}}
\pysigstartsignatures
\pysiglinewithargsret{\sphinxbfcode{\sphinxupquote{CreateCatalogueActive}}}{}{}
\pysigstopsignatures
\sphinxAtStartPar
Merge the JSR and SpaceTrack active satellite catalogues.

\sphinxAtStartPar
The resulting merged catalogue is stored in an attribute named CurrentCatalogueDF.
\begin{quote}\begin{description}
\sphinxlineitem{Raises}\begin{itemize}
\item {} 
\sphinxAtStartPar
\sphinxstyleliteralstrong{\sphinxupquote{IOError}} – If file paths are not found or unreadable.

\item {} 
\sphinxAtStartPar
\sphinxstyleliteralstrong{\sphinxupquote{ValueError}} – If there’s an issue with the data content or format.

\end{itemize}

\sphinxlineitem{Returns}
\sphinxAtStartPar
None. But exports a CSV file with merged list of space objects to ‘src/fspsim/data/external/active\_jsr\_spacetrack.csv’.

\sphinxlineitem{Return type}
\sphinxAtStartPar
None

\end{description}\end{quote}

\end{fulllineitems}

\index{CreateCatalogueAll() (fspsim.utils.SpaceCatalogue.SpaceCatalogue method)@\spxentry{CreateCatalogueAll()}\spxextra{fspsim.utils.SpaceCatalogue.SpaceCatalogue method}}

\begin{fulllineitems}
\phantomsection\label{\detokenize{fspsim.utils:fspsim.utils.SpaceCatalogue.SpaceCatalogue.CreateCatalogueAll}}
\pysigstartsignatures
\pysiglinewithargsret{\sphinxbfcode{\sphinxupquote{CreateCatalogueAll}}}{}{}
\pysigstopsignatures
\sphinxAtStartPar
Merge the JSR catalogue with the SpaceTrack catalogue.

\sphinxAtStartPar
The function prioritizes SpaceTrack’s data in case of conflicting information.
The resulting merged catalogue is stored in an attribute named CurrentCatalogueDF.
\begin{quote}\begin{description}
\sphinxlineitem{Raises}\begin{itemize}
\item {} 
\sphinxAtStartPar
\sphinxstyleliteralstrong{\sphinxupquote{IOError}} – If file paths are not found or unreadable.

\item {} 
\sphinxAtStartPar
\sphinxstyleliteralstrong{\sphinxupquote{ValueError}} – If there’s an issue with the data content or format.

\end{itemize}

\sphinxlineitem{Returns}
\sphinxAtStartPar
None. But exports a CSV file of merged space catalogue of all tracked objects to ‘src/fspsim/data/catalogue/All\_catalogue\_latest.txt’.

\sphinxlineitem{Return type}
\sphinxAtStartPar
None

\sphinxlineitem{Note}
\sphinxAtStartPar
In the case of missing data for the simulation, the function will drop the rows that do not have the required data. At present this results in a couple thousands of objects being dropped.

\end{description}\end{quote}

\end{fulllineitems}

\index{DownloadJSRCatalogueIfNewer() (fspsim.utils.SpaceCatalogue.SpaceCatalogue method)@\spxentry{DownloadJSRCatalogueIfNewer()}\spxextra{fspsim.utils.SpaceCatalogue.SpaceCatalogue method}}

\begin{fulllineitems}
\phantomsection\label{\detokenize{fspsim.utils:fspsim.utils.SpaceCatalogue.SpaceCatalogue.DownloadJSRCatalogueIfNewer}}
\pysigstartsignatures
\pysiglinewithargsret{\sphinxbfcode{\sphinxupquote{DownloadJSRCatalogueIfNewer}}}{\sphinxparam{\DUrole{n}{local\_path}}\sphinxparamcomma \sphinxparam{\DUrole{n}{url}}}{}
\pysigstopsignatures
\sphinxAtStartPar
Download a file from a given URL if it is newer than the local file.
\begin{quote}\begin{description}
\sphinxlineitem{Parameters}\begin{itemize}
\item {} 
\sphinxAtStartPar
\sphinxstyleliteralstrong{\sphinxupquote{local\_path}} (\sphinxstyleliteralemphasis{\sphinxupquote{str}}) – Local file path.

\item {} 
\sphinxAtStartPar
\sphinxstyleliteralstrong{\sphinxupquote{url}} (\sphinxstyleliteralemphasis{\sphinxupquote{str}}) – URL of the file to be downloaded.

\end{itemize}

\sphinxlineitem{Returns}
\sphinxAtStartPar
None. Downloads the file if newer.

\sphinxlineitem{Return type}
\sphinxAtStartPar
None

\end{description}\end{quote}

\end{fulllineitems}

\index{PullCatalogueJSR() (fspsim.utils.SpaceCatalogue.SpaceCatalogue method)@\spxentry{PullCatalogueJSR()}\spxextra{fspsim.utils.SpaceCatalogue.SpaceCatalogue method}}

\begin{fulllineitems}
\phantomsection\label{\detokenize{fspsim.utils:fspsim.utils.SpaceCatalogue.SpaceCatalogue.PullCatalogueJSR}}
\pysigstartsignatures
\pysiglinewithargsret{\sphinxbfcode{\sphinxupquote{PullCatalogueJSR}}}{\sphinxparam{\DUrole{n}{external\_dir}}}{}
\pysigstopsignatures
\sphinxAtStartPar
Update the JSR catalogue files by downloading them if a newer version is available online.
\begin{quote}\begin{description}
\sphinxlineitem{Parameters}
\sphinxAtStartPar
\sphinxstyleliteralstrong{\sphinxupquote{external\_dir}} (\sphinxstyleliteralemphasis{\sphinxupquote{str}}) – The directory path where the files will be saved.

\sphinxlineitem{Returns}
\sphinxAtStartPar
None. Updates the JSR catalogue files.

\sphinxlineitem{Return type}
\sphinxAtStartPar
None

\end{description}\end{quote}

\end{fulllineitems}

\index{PullCatalogueSpaceTrack() (fspsim.utils.SpaceCatalogue.SpaceCatalogue method)@\spxentry{PullCatalogueSpaceTrack()}\spxextra{fspsim.utils.SpaceCatalogue.SpaceCatalogue method}}

\begin{fulllineitems}
\phantomsection\label{\detokenize{fspsim.utils:fspsim.utils.SpaceCatalogue.SpaceCatalogue.PullCatalogueSpaceTrack}}
\pysigstartsignatures
\pysiglinewithargsret{\sphinxbfcode{\sphinxupquote{PullCatalogueSpaceTrack}}}{\sphinxparam{\DUrole{n}{external\_dir}}}{}
\pysigstopsignatures
\sphinxAtStartPar
Download the entire SpaceTrack catalogue.

\sphinxAtStartPar
This method requires SpaceTrack login credentials stored in a ‘.env’ file.
\begin{quote}\begin{description}
\sphinxlineitem{Parameters}
\sphinxAtStartPar
\sphinxstyleliteralstrong{\sphinxupquote{external\_dir}} (\sphinxstyleliteralemphasis{\sphinxupquote{str}}) – The directory path where the SpaceTrack catalogue file will be saved.

\sphinxlineitem{Raises}
\sphinxAtStartPar
\sphinxstyleliteralstrong{\sphinxupquote{Exception}} – If there’s an issue with the login credentials or fetching data.

\sphinxlineitem{Returns}
\sphinxAtStartPar
None. Saves the SpaceTrack catalogue in a JSON file.

\sphinxlineitem{Return type}
\sphinxAtStartPar
None

\end{description}\end{quote}

\end{fulllineitems}

\index{load\_from\_file() (fspsim.utils.SpaceCatalogue.SpaceCatalogue class method)@\spxentry{load\_from\_file()}\spxextra{fspsim.utils.SpaceCatalogue.SpaceCatalogue class method}}

\begin{fulllineitems}
\phantomsection\label{\detokenize{fspsim.utils:fspsim.utils.SpaceCatalogue.SpaceCatalogue.load_from_file}}
\pysigstartsignatures
\pysiglinewithargsret{\sphinxbfcode{\sphinxupquote{classmethod\DUrole{w}{ }}}\sphinxbfcode{\sphinxupquote{load\_from\_file}}}{\sphinxparam{\DUrole{n}{file\_path}}}{}
\pysigstopsignatures
\end{fulllineitems}


\end{fulllineitems}

\index{check\_json\_file() (in module fspsim.utils.SpaceCatalogue)@\spxentry{check\_json\_file()}\spxextra{in module fspsim.utils.SpaceCatalogue}}

\begin{fulllineitems}
\phantomsection\label{\detokenize{fspsim.utils:fspsim.utils.SpaceCatalogue.check_json_file}}
\pysigstartsignatures
\pysiglinewithargsret{\sphinxcode{\sphinxupquote{fspsim.utils.SpaceCatalogue.}}\sphinxbfcode{\sphinxupquote{check\_json\_file}}}{\sphinxparam{\DUrole{n}{json\_data}}}{}
\pysigstopsignatures
\sphinxAtStartPar
Checks the validity of the provided JSON content.
\begin{quote}\begin{description}
\sphinxlineitem{Parameters}
\sphinxAtStartPar
\sphinxstyleliteralstrong{\sphinxupquote{json\_data}} (\sphinxstyleliteralemphasis{\sphinxupquote{dict}}) – Dictionary parsed from a JSON file that needs to be checked.

\sphinxlineitem{Raises}\begin{itemize}
\item {} 
\sphinxAtStartPar
\sphinxstyleliteralstrong{\sphinxupquote{KeyError}} – If an expected key is not found in the JSON content.

\item {} 
\sphinxAtStartPar
\sphinxstyleliteralstrong{\sphinxupquote{ValueError}} – If the value of a key is not in the list of valid options.

\end{itemize}

\end{description}\end{quote}

\end{fulllineitems}

\index{dump\_pickle() (in module fspsim.utils.SpaceCatalogue)@\spxentry{dump\_pickle()}\spxextra{in module fspsim.utils.SpaceCatalogue}}

\begin{fulllineitems}
\phantomsection\label{\detokenize{fspsim.utils:fspsim.utils.SpaceCatalogue.dump_pickle}}
\pysigstartsignatures
\pysiglinewithargsret{\sphinxcode{\sphinxupquote{fspsim.utils.SpaceCatalogue.}}\sphinxbfcode{\sphinxupquote{dump\_pickle}}}{\sphinxparam{\DUrole{n}{file\_path}}\sphinxparamcomma \sphinxparam{\DUrole{n}{data}}}{}
\pysigstopsignatures
\sphinxAtStartPar
Saves provided data to a pickle file.
\begin{quote}\begin{description}
\sphinxlineitem{Parameters}\begin{itemize}
\item {} 
\sphinxAtStartPar
\sphinxstyleliteralstrong{\sphinxupquote{file\_path}} (\sphinxstyleliteralemphasis{\sphinxupquote{str}}) – Relative path where the data will be saved in pickle format.

\item {} 
\sphinxAtStartPar
\sphinxstyleliteralstrong{\sphinxupquote{data}} (\sphinxstyleliteralemphasis{\sphinxupquote{Any}}) – Data to be saved in the pickle file.

\end{itemize}

\end{description}\end{quote}

\end{fulllineitems}

\index{get\_path() (in module fspsim.utils.SpaceCatalogue)@\spxentry{get\_path()}\spxextra{in module fspsim.utils.SpaceCatalogue}}

\begin{fulllineitems}
\phantomsection\label{\detokenize{fspsim.utils:fspsim.utils.SpaceCatalogue.get_path}}
\pysigstartsignatures
\pysiglinewithargsret{\sphinxcode{\sphinxupquote{fspsim.utils.SpaceCatalogue.}}\sphinxbfcode{\sphinxupquote{get\_path}}}{\sphinxparam{\DUrole{o}{*}\DUrole{n}{args}}}{}
\pysigstopsignatures
\sphinxAtStartPar
Constructs a full path by joining the current working directory with the provided arguments.
\begin{quote}\begin{description}
\sphinxlineitem{Parameters}
\sphinxAtStartPar
\sphinxstyleliteralstrong{\sphinxupquote{args}} (\sphinxstyleliteralemphasis{\sphinxupquote{str}}) – Arguments that constitute the relative path.

\sphinxlineitem{Returns}
\sphinxAtStartPar
The full path generated by joining the current directory with the provided arguments.

\sphinxlineitem{Return type}
\sphinxAtStartPar
str

\end{description}\end{quote}

\end{fulllineitems}

\index{load\_pickle() (in module fspsim.utils.SpaceCatalogue)@\spxentry{load\_pickle()}\spxextra{in module fspsim.utils.SpaceCatalogue}}

\begin{fulllineitems}
\phantomsection\label{\detokenize{fspsim.utils:fspsim.utils.SpaceCatalogue.load_pickle}}
\pysigstartsignatures
\pysiglinewithargsret{\sphinxcode{\sphinxupquote{fspsim.utils.SpaceCatalogue.}}\sphinxbfcode{\sphinxupquote{load\_pickle}}}{\sphinxparam{\DUrole{n}{file\_path}}}{}
\pysigstopsignatures
\sphinxAtStartPar
Loads and returns data from a pickle file.
\begin{quote}\begin{description}
\sphinxlineitem{Parameters}
\sphinxAtStartPar
\sphinxstyleliteralstrong{\sphinxupquote{file\_path}} (\sphinxstyleliteralemphasis{\sphinxupquote{str}}) – Relative path to the pickle file to be loaded.

\sphinxlineitem{Returns}
\sphinxAtStartPar
Data loaded from the pickle file.

\sphinxlineitem{Return type}
\sphinxAtStartPar
Any

\end{description}\end{quote}

\end{fulllineitems}



\paragraph{fspsim.utils.SpaceObject module}
\label{\detokenize{fspsim.utils:module-fspsim.utils.SpaceObject}}\label{\detokenize{fspsim.utils:fspsim-utils-spaceobject-module}}\index{module@\spxentry{module}!fspsim.utils.SpaceObject@\spxentry{fspsim.utils.SpaceObject}}\index{fspsim.utils.SpaceObject@\spxentry{fspsim.utils.SpaceObject}!module@\spxentry{module}}\index{ObjectType (class in fspsim.utils.SpaceObject)@\spxentry{ObjectType}\spxextra{class in fspsim.utils.SpaceObject}}

\begin{fulllineitems}
\phantomsection\label{\detokenize{fspsim.utils:fspsim.utils.SpaceObject.ObjectType}}
\pysigstartsignatures
\pysiglinewithargsret{\sphinxbfcode{\sphinxupquote{class\DUrole{w}{ }}}\sphinxcode{\sphinxupquote{fspsim.utils.SpaceObject.}}\sphinxbfcode{\sphinxupquote{ObjectType}}}{\sphinxparam{\DUrole{n}{value}}\sphinxparamcomma \sphinxparam{\DUrole{n}{names}\DUrole{o}{=}\DUrole{default_value}{None}}\sphinxparamcomma \sphinxparam{\DUrole{o}{*}}\sphinxparamcomma \sphinxparam{\DUrole{n}{module}\DUrole{o}{=}\DUrole{default_value}{None}}\sphinxparamcomma \sphinxparam{\DUrole{n}{qualname}\DUrole{o}{=}\DUrole{default_value}{None}}\sphinxparamcomma \sphinxparam{\DUrole{n}{type}\DUrole{o}{=}\DUrole{default_value}{None}}\sphinxparamcomma \sphinxparam{\DUrole{n}{start}\DUrole{o}{=}\DUrole{default_value}{1}}\sphinxparamcomma \sphinxparam{\DUrole{n}{boundary}\DUrole{o}{=}\DUrole{default_value}{None}}}{}
\pysigstopsignatures
\sphinxAtStartPar
Bases: \sphinxcode{\sphinxupquote{Enum}}
\index{DEB (fspsim.utils.SpaceObject.ObjectType attribute)@\spxentry{DEB}\spxextra{fspsim.utils.SpaceObject.ObjectType attribute}}

\begin{fulllineitems}
\phantomsection\label{\detokenize{fspsim.utils:fspsim.utils.SpaceObject.ObjectType.DEB}}
\pysigstartsignatures
\pysigline{\sphinxbfcode{\sphinxupquote{DEB}}\sphinxbfcode{\sphinxupquote{\DUrole{w}{ }\DUrole{p}{=}\DUrole{w}{ }'DEB'}}}
\pysigstopsignatures
\end{fulllineitems}

\index{OTHER (fspsim.utils.SpaceObject.ObjectType attribute)@\spxentry{OTHER}\spxextra{fspsim.utils.SpaceObject.ObjectType attribute}}

\begin{fulllineitems}
\phantomsection\label{\detokenize{fspsim.utils:fspsim.utils.SpaceObject.ObjectType.OTHER}}
\pysigstartsignatures
\pysigline{\sphinxbfcode{\sphinxupquote{OTHER}}\sphinxbfcode{\sphinxupquote{\DUrole{w}{ }\DUrole{p}{=}\DUrole{w}{ }'?'}}}
\pysigstopsignatures
\end{fulllineitems}

\index{PAYLOAD (fspsim.utils.SpaceObject.ObjectType attribute)@\spxentry{PAYLOAD}\spxextra{fspsim.utils.SpaceObject.ObjectType attribute}}

\begin{fulllineitems}
\phantomsection\label{\detokenize{fspsim.utils:fspsim.utils.SpaceObject.ObjectType.PAYLOAD}}
\pysigstartsignatures
\pysigline{\sphinxbfcode{\sphinxupquote{PAYLOAD}}\sphinxbfcode{\sphinxupquote{\DUrole{w}{ }\DUrole{p}{=}\DUrole{w}{ }'PAY'}}}
\pysigstopsignatures
\end{fulllineitems}

\index{ROCKET\_BODY (fspsim.utils.SpaceObject.ObjectType attribute)@\spxentry{ROCKET\_BODY}\spxextra{fspsim.utils.SpaceObject.ObjectType attribute}}

\begin{fulllineitems}
\phantomsection\label{\detokenize{fspsim.utils:fspsim.utils.SpaceObject.ObjectType.ROCKET_BODY}}
\pysigstartsignatures
\pysigline{\sphinxbfcode{\sphinxupquote{ROCKET\_BODY}}\sphinxbfcode{\sphinxupquote{\DUrole{w}{ }\DUrole{p}{=}\DUrole{w}{ }'R/B'}}}
\pysigstopsignatures
\end{fulllineitems}

\index{UNKNOWN (fspsim.utils.SpaceObject.ObjectType attribute)@\spxentry{UNKNOWN}\spxextra{fspsim.utils.SpaceObject.ObjectType attribute}}

\begin{fulllineitems}
\phantomsection\label{\detokenize{fspsim.utils:fspsim.utils.SpaceObject.ObjectType.UNKNOWN}}
\pysigstartsignatures
\pysigline{\sphinxbfcode{\sphinxupquote{UNKNOWN}}\sphinxbfcode{\sphinxupquote{\DUrole{w}{ }\DUrole{p}{=}\DUrole{w}{ }'UNK'}}}
\pysigstopsignatures
\end{fulllineitems}


\end{fulllineitems}

\index{OperationalStatus (class in fspsim.utils.SpaceObject)@\spxentry{OperationalStatus}\spxextra{class in fspsim.utils.SpaceObject}}

\begin{fulllineitems}
\phantomsection\label{\detokenize{fspsim.utils:fspsim.utils.SpaceObject.OperationalStatus}}
\pysigstartsignatures
\pysiglinewithargsret{\sphinxbfcode{\sphinxupquote{class\DUrole{w}{ }}}\sphinxcode{\sphinxupquote{fspsim.utils.SpaceObject.}}\sphinxbfcode{\sphinxupquote{OperationalStatus}}}{\sphinxparam{\DUrole{n}{value}}\sphinxparamcomma \sphinxparam{\DUrole{n}{names}\DUrole{o}{=}\DUrole{default_value}{None}}\sphinxparamcomma \sphinxparam{\DUrole{o}{*}}\sphinxparamcomma \sphinxparam{\DUrole{n}{module}\DUrole{o}{=}\DUrole{default_value}{None}}\sphinxparamcomma \sphinxparam{\DUrole{n}{qualname}\DUrole{o}{=}\DUrole{default_value}{None}}\sphinxparamcomma \sphinxparam{\DUrole{n}{type}\DUrole{o}{=}\DUrole{default_value}{None}}\sphinxparamcomma \sphinxparam{\DUrole{n}{start}\DUrole{o}{=}\DUrole{default_value}{1}}\sphinxparamcomma \sphinxparam{\DUrole{n}{boundary}\DUrole{o}{=}\DUrole{default_value}{None}}}{}
\pysigstopsignatures
\sphinxAtStartPar
Bases: \sphinxcode{\sphinxupquote{Enum}}
\index{B (fspsim.utils.SpaceObject.OperationalStatus attribute)@\spxentry{B}\spxextra{fspsim.utils.SpaceObject.OperationalStatus attribute}}

\begin{fulllineitems}
\phantomsection\label{\detokenize{fspsim.utils:fspsim.utils.SpaceObject.OperationalStatus.B}}
\pysigstartsignatures
\pysigline{\sphinxbfcode{\sphinxupquote{B}}\sphinxbfcode{\sphinxupquote{\DUrole{w}{ }\DUrole{p}{=}\DUrole{w}{ }'B'}}}
\pysigstopsignatures
\end{fulllineitems}

\index{D (fspsim.utils.SpaceObject.OperationalStatus attribute)@\spxentry{D}\spxextra{fspsim.utils.SpaceObject.OperationalStatus attribute}}

\begin{fulllineitems}
\phantomsection\label{\detokenize{fspsim.utils:fspsim.utils.SpaceObject.OperationalStatus.D}}
\pysigstartsignatures
\pysigline{\sphinxbfcode{\sphinxupquote{D}}\sphinxbfcode{\sphinxupquote{\DUrole{w}{ }\DUrole{p}{=}\DUrole{w}{ }'D'}}}
\pysigstopsignatures
\end{fulllineitems}

\index{NEGATIVE (fspsim.utils.SpaceObject.OperationalStatus attribute)@\spxentry{NEGATIVE}\spxextra{fspsim.utils.SpaceObject.OperationalStatus attribute}}

\begin{fulllineitems}
\phantomsection\label{\detokenize{fspsim.utils:fspsim.utils.SpaceObject.OperationalStatus.NEGATIVE}}
\pysigstartsignatures
\pysigline{\sphinxbfcode{\sphinxupquote{NEGATIVE}}\sphinxbfcode{\sphinxupquote{\DUrole{w}{ }\DUrole{p}{=}\DUrole{w}{ }'\sphinxhyphen{}'}}}
\pysigstopsignatures
\end{fulllineitems}

\index{PARTIAL (fspsim.utils.SpaceObject.OperationalStatus attribute)@\spxentry{PARTIAL}\spxextra{fspsim.utils.SpaceObject.OperationalStatus attribute}}

\begin{fulllineitems}
\phantomsection\label{\detokenize{fspsim.utils:fspsim.utils.SpaceObject.OperationalStatus.PARTIAL}}
\pysigstartsignatures
\pysigline{\sphinxbfcode{\sphinxupquote{PARTIAL}}\sphinxbfcode{\sphinxupquote{\DUrole{w}{ }\DUrole{p}{=}\DUrole{w}{ }'P'}}}
\pysigstopsignatures
\end{fulllineitems}

\index{POSITIVE (fspsim.utils.SpaceObject.OperationalStatus attribute)@\spxentry{POSITIVE}\spxextra{fspsim.utils.SpaceObject.OperationalStatus attribute}}

\begin{fulllineitems}
\phantomsection\label{\detokenize{fspsim.utils:fspsim.utils.SpaceObject.OperationalStatus.POSITIVE}}
\pysigstartsignatures
\pysigline{\sphinxbfcode{\sphinxupquote{POSITIVE}}\sphinxbfcode{\sphinxupquote{\DUrole{w}{ }\DUrole{p}{=}\DUrole{w}{ }'+'}}}
\pysigstopsignatures
\end{fulllineitems}

\index{S (fspsim.utils.SpaceObject.OperationalStatus attribute)@\spxentry{S}\spxextra{fspsim.utils.SpaceObject.OperationalStatus attribute}}

\begin{fulllineitems}
\phantomsection\label{\detokenize{fspsim.utils:fspsim.utils.SpaceObject.OperationalStatus.S}}
\pysigstartsignatures
\pysigline{\sphinxbfcode{\sphinxupquote{S}}\sphinxbfcode{\sphinxupquote{\DUrole{w}{ }\DUrole{p}{=}\DUrole{w}{ }'S'}}}
\pysigstopsignatures
\end{fulllineitems}

\index{UNKNOWN (fspsim.utils.SpaceObject.OperationalStatus attribute)@\spxentry{UNKNOWN}\spxextra{fspsim.utils.SpaceObject.OperationalStatus attribute}}

\begin{fulllineitems}
\phantomsection\label{\detokenize{fspsim.utils:fspsim.utils.SpaceObject.OperationalStatus.UNKNOWN}}
\pysigstartsignatures
\pysigline{\sphinxbfcode{\sphinxupquote{UNKNOWN}}\sphinxbfcode{\sphinxupquote{\DUrole{w}{ }\DUrole{p}{=}\DUrole{w}{ }'?'}}}
\pysigstopsignatures
\end{fulllineitems}

\index{X (fspsim.utils.SpaceObject.OperationalStatus attribute)@\spxentry{X}\spxextra{fspsim.utils.SpaceObject.OperationalStatus attribute}}

\begin{fulllineitems}
\phantomsection\label{\detokenize{fspsim.utils:fspsim.utils.SpaceObject.OperationalStatus.X}}
\pysigstartsignatures
\pysigline{\sphinxbfcode{\sphinxupquote{X}}\sphinxbfcode{\sphinxupquote{\DUrole{w}{ }\DUrole{p}{=}\DUrole{w}{ }'X'}}}
\pysigstopsignatures
\end{fulllineitems}


\end{fulllineitems}

\index{SpaceObject (class in fspsim.utils.SpaceObject)@\spxentry{SpaceObject}\spxextra{class in fspsim.utils.SpaceObject}}

\begin{fulllineitems}
\phantomsection\label{\detokenize{fspsim.utils:fspsim.utils.SpaceObject.SpaceObject}}
\pysigstartsignatures
\pysiglinewithargsret{\sphinxbfcode{\sphinxupquote{class\DUrole{w}{ }}}\sphinxcode{\sphinxupquote{fspsim.utils.SpaceObject.}}\sphinxbfcode{\sphinxupquote{SpaceObject}}}{\sphinxparam{\DUrole{n}{rso\_name}\DUrole{o}{=}\DUrole{default_value}{None}}\sphinxparamcomma \sphinxparam{\DUrole{n}{rso\_type}\DUrole{o}{=}\DUrole{default_value}{None}}\sphinxparamcomma \sphinxparam{\DUrole{n}{payload\_operational\_status}\DUrole{o}{=}\DUrole{default_value}{None}}\sphinxparamcomma \sphinxparam{\DUrole{n}{application}\DUrole{o}{=}\DUrole{default_value}{None}}\sphinxparamcomma \sphinxparam{\DUrole{n}{source}\DUrole{o}{=}\DUrole{default_value}{None}}\sphinxparamcomma \sphinxparam{\DUrole{n}{launch\_site}\DUrole{o}{=}\DUrole{default_value}{None}}\sphinxparamcomma \sphinxparam{\DUrole{n}{mass}\DUrole{o}{=}\DUrole{default_value}{None}}\sphinxparamcomma \sphinxparam{\DUrole{n}{maneuverable}\DUrole{o}{=}\DUrole{default_value}{False}}\sphinxparamcomma \sphinxparam{\DUrole{n}{spin\_stabilized}\DUrole{o}{=}\DUrole{default_value}{False}}\sphinxparamcomma \sphinxparam{\DUrole{n}{object\_type}\DUrole{o}{=}\DUrole{default_value}{None}}\sphinxparamcomma \sphinxparam{\DUrole{n}{apogee}\DUrole{o}{=}\DUrole{default_value}{None}}\sphinxparamcomma \sphinxparam{\DUrole{n}{perigee}\DUrole{o}{=}\DUrole{default_value}{None}}\sphinxparamcomma \sphinxparam{\DUrole{n}{characteristic\_area}\DUrole{o}{=}\DUrole{default_value}{None}}\sphinxparamcomma \sphinxparam{\DUrole{n}{characteristic\_length}\DUrole{o}{=}\DUrole{default_value}{None}}\sphinxparamcomma \sphinxparam{\DUrole{n}{propulsion\_type}\DUrole{o}{=}\DUrole{default_value}{None}}\sphinxparamcomma \sphinxparam{\DUrole{n}{epoch}\DUrole{o}{=}\DUrole{default_value}{None}}\sphinxparamcomma \sphinxparam{\DUrole{n}{sma}\DUrole{o}{=}\DUrole{default_value}{None}}\sphinxparamcomma \sphinxparam{\DUrole{n}{inc}\DUrole{o}{=}\DUrole{default_value}{None}}\sphinxparamcomma \sphinxparam{\DUrole{n}{argp}\DUrole{o}{=}\DUrole{default_value}{None}}\sphinxparamcomma \sphinxparam{\DUrole{n}{raan}\DUrole{o}{=}\DUrole{default_value}{None}}\sphinxparamcomma \sphinxparam{\DUrole{n}{tran}\DUrole{o}{=}\DUrole{default_value}{None}}\sphinxparamcomma \sphinxparam{\DUrole{n}{eccentricity}\DUrole{o}{=}\DUrole{default_value}{None}}\sphinxparamcomma \sphinxparam{\DUrole{n}{operator}\DUrole{o}{=}\DUrole{default_value}{None}}\sphinxparamcomma \sphinxparam{\DUrole{n}{launch\_date}\DUrole{o}{=}\DUrole{default_value}{None}}\sphinxparamcomma \sphinxparam{\DUrole{n}{decay\_date}\DUrole{o}{=}\DUrole{default_value}{None}}\sphinxparamcomma \sphinxparam{\DUrole{n}{tle}\DUrole{o}{=}\DUrole{default_value}{None}}\sphinxparamcomma \sphinxparam{\DUrole{n}{station\_keeping}\DUrole{o}{=}\DUrole{default_value}{None}}\sphinxparamcomma \sphinxparam{\DUrole{n}{orbit\_source}\DUrole{o}{=}\DUrole{default_value}{None}}}{}
\pysigstopsignatures
\sphinxAtStartPar
Bases: \sphinxcode{\sphinxupquote{object}}

\sphinxAtStartPar
A representation of a space object detailing its orbital and physical properties.

\sphinxAtStartPar
This class provides comprehensive details about the space object, including its launch date,
operational status, object type, application, and a variety of other orbital parameters.
\index{launch\_date (fspsim.utils.SpaceObject.SpaceObject attribute)@\spxentry{launch\_date}\spxextra{fspsim.utils.SpaceObject.SpaceObject attribute}}

\begin{fulllineitems}
\phantomsection\label{\detokenize{fspsim.utils:fspsim.utils.SpaceObject.SpaceObject.launch_date}}
\pysigstartsignatures
\pysigline{\sphinxbfcode{\sphinxupquote{launch\_date}}}
\pysigstopsignatures
\sphinxAtStartPar
The date when the object was launched.
\begin{quote}\begin{description}
\sphinxlineitem{Type}
\sphinxAtStartPar
datetime

\end{description}\end{quote}

\end{fulllineitems}

\index{decay\_date (fspsim.utils.SpaceObject.SpaceObject attribute)@\spxentry{decay\_date}\spxextra{fspsim.utils.SpaceObject.SpaceObject attribute}}

\begin{fulllineitems}
\phantomsection\label{\detokenize{fspsim.utils:fspsim.utils.SpaceObject.SpaceObject.decay_date}}
\pysigstartsignatures
\pysigline{\sphinxbfcode{\sphinxupquote{decay\_date}}}
\pysigstopsignatures
\sphinxAtStartPar
The date when the object is expected to decay.
\begin{quote}\begin{description}
\sphinxlineitem{Type}
\sphinxAtStartPar
datetime

\end{description}\end{quote}

\end{fulllineitems}

\index{rso\_name (fspsim.utils.SpaceObject.SpaceObject attribute)@\spxentry{rso\_name}\spxextra{fspsim.utils.SpaceObject.SpaceObject attribute}}

\begin{fulllineitems}
\phantomsection\label{\detokenize{fspsim.utils:fspsim.utils.SpaceObject.SpaceObject.rso_name}}
\pysigstartsignatures
\pysigline{\sphinxbfcode{\sphinxupquote{rso\_name}}}
\pysigstopsignatures
\sphinxAtStartPar
Name of the Resident Space Object (RSO).
\begin{quote}\begin{description}
\sphinxlineitem{Type}
\sphinxAtStartPar
str

\end{description}\end{quote}

\end{fulllineitems}

\index{rso\_type (fspsim.utils.SpaceObject.SpaceObject attribute)@\spxentry{rso\_type}\spxextra{fspsim.utils.SpaceObject.SpaceObject attribute}}

\begin{fulllineitems}
\phantomsection\label{\detokenize{fspsim.utils:fspsim.utils.SpaceObject.SpaceObject.rso_type}}
\pysigstartsignatures
\pysigline{\sphinxbfcode{\sphinxupquote{rso\_type}}}
\pysigstopsignatures
\sphinxAtStartPar
Type of the RSO.
\begin{quote}\begin{description}
\sphinxlineitem{Type}
\sphinxAtStartPar
str

\end{description}\end{quote}

\end{fulllineitems}

\index{payload\_operational\_status (fspsim.utils.SpaceObject.SpaceObject attribute)@\spxentry{payload\_operational\_status}\spxextra{fspsim.utils.SpaceObject.SpaceObject attribute}}

\begin{fulllineitems}
\phantomsection\label{\detokenize{fspsim.utils:fspsim.utils.SpaceObject.SpaceObject.payload_operational_status}}
\pysigstartsignatures
\pysigline{\sphinxbfcode{\sphinxupquote{payload\_operational\_status}}}
\pysigstopsignatures
\sphinxAtStartPar
Operational status of the payload.
\begin{quote}\begin{description}
\sphinxlineitem{Type}
\sphinxAtStartPar
str

\end{description}\end{quote}

\end{fulllineitems}

\index{object\_type (fspsim.utils.SpaceObject.SpaceObject attribute)@\spxentry{object\_type}\spxextra{fspsim.utils.SpaceObject.SpaceObject attribute}}

\begin{fulllineitems}
\phantomsection\label{\detokenize{fspsim.utils:fspsim.utils.SpaceObject.SpaceObject.object_type}}
\pysigstartsignatures
\pysigline{\sphinxbfcode{\sphinxupquote{object\_type}}}
\pysigstopsignatures
\sphinxAtStartPar
Type of the object. Can be ‘DEB’, ‘PAY’, ‘R/B’, ‘UNK’, or ‘?’.
\begin{quote}\begin{description}
\sphinxlineitem{Type}
\sphinxAtStartPar
str

\end{description}\end{quote}

\end{fulllineitems}

\index{application (fspsim.utils.SpaceObject.SpaceObject attribute)@\spxentry{application}\spxextra{fspsim.utils.SpaceObject.SpaceObject attribute}}

\begin{fulllineitems}
\phantomsection\label{\detokenize{fspsim.utils:fspsim.utils.SpaceObject.SpaceObject.application}}
\pysigstartsignatures
\pysigline{\sphinxbfcode{\sphinxupquote{application}}}
\pysigstopsignatures
\sphinxAtStartPar
Application purpose of the space object.
\begin{quote}\begin{description}
\sphinxlineitem{Type}
\sphinxAtStartPar
str

\end{description}\end{quote}

\end{fulllineitems}

\index{operator (fspsim.utils.SpaceObject.SpaceObject attribute)@\spxentry{operator}\spxextra{fspsim.utils.SpaceObject.SpaceObject attribute}}

\begin{fulllineitems}
\phantomsection\label{\detokenize{fspsim.utils:fspsim.utils.SpaceObject.SpaceObject.operator}}
\pysigstartsignatures
\pysigline{\sphinxbfcode{\sphinxupquote{operator}}}
\pysigstopsignatures
\sphinxAtStartPar
The operator or owner of the object.
\begin{quote}\begin{description}
\sphinxlineitem{Type}
\sphinxAtStartPar
str

\end{description}\end{quote}

\end{fulllineitems}

\index{characteristic\_length (fspsim.utils.SpaceObject.SpaceObject attribute)@\spxentry{characteristic\_length}\spxextra{fspsim.utils.SpaceObject.SpaceObject attribute}}

\begin{fulllineitems}
\phantomsection\label{\detokenize{fspsim.utils:fspsim.utils.SpaceObject.SpaceObject.characteristic_length}}
\pysigstartsignatures
\pysigline{\sphinxbfcode{\sphinxupquote{characteristic\_length}}}
\pysigstopsignatures
\sphinxAtStartPar
The characteristic length of the object.
\begin{quote}\begin{description}
\sphinxlineitem{Type}
\sphinxAtStartPar
float

\end{description}\end{quote}

\end{fulllineitems}

\index{characteristic\_area (fspsim.utils.SpaceObject.SpaceObject attribute)@\spxentry{characteristic\_area}\spxextra{fspsim.utils.SpaceObject.SpaceObject attribute}}

\begin{fulllineitems}
\phantomsection\label{\detokenize{fspsim.utils:fspsim.utils.SpaceObject.SpaceObject.characteristic_area}}
\pysigstartsignatures
\pysigline{\sphinxbfcode{\sphinxupquote{characteristic\_area}}}
\pysigstopsignatures
\sphinxAtStartPar
The characteristic area of the object.
\begin{quote}\begin{description}
\sphinxlineitem{Type}
\sphinxAtStartPar
float

\end{description}\end{quote}

\end{fulllineitems}

\index{mass (fspsim.utils.SpaceObject.SpaceObject attribute)@\spxentry{mass}\spxextra{fspsim.utils.SpaceObject.SpaceObject attribute}}

\begin{fulllineitems}
\phantomsection\label{\detokenize{fspsim.utils:fspsim.utils.SpaceObject.SpaceObject.mass}}
\pysigstartsignatures
\pysigline{\sphinxbfcode{\sphinxupquote{mass}}}
\pysigstopsignatures
\sphinxAtStartPar
The mass of the object.
\begin{quote}\begin{description}
\sphinxlineitem{Type}
\sphinxAtStartPar
float

\end{description}\end{quote}

\end{fulllineitems}

\index{source (fspsim.utils.SpaceObject.SpaceObject attribute)@\spxentry{source}\spxextra{fspsim.utils.SpaceObject.SpaceObject attribute}}

\begin{fulllineitems}
\phantomsection\label{\detokenize{fspsim.utils:fspsim.utils.SpaceObject.SpaceObject.source}}
\pysigstartsignatures
\pysigline{\sphinxbfcode{\sphinxupquote{source}}}
\pysigstopsignatures
\sphinxAtStartPar
Source or country of origin of the object.
\begin{quote}\begin{description}
\sphinxlineitem{Type}
\sphinxAtStartPar
str

\end{description}\end{quote}

\end{fulllineitems}

\index{launch\_site (fspsim.utils.SpaceObject.SpaceObject attribute)@\spxentry{launch\_site}\spxextra{fspsim.utils.SpaceObject.SpaceObject attribute}}

\begin{fulllineitems}
\phantomsection\label{\detokenize{fspsim.utils:fspsim.utils.SpaceObject.SpaceObject.launch_site}}
\pysigstartsignatures
\pysigline{\sphinxbfcode{\sphinxupquote{launch\_site}}}
\pysigstopsignatures
\sphinxAtStartPar
The site where the object was launched from.
\begin{quote}\begin{description}
\sphinxlineitem{Type}
\sphinxAtStartPar
str

\end{description}\end{quote}

\end{fulllineitems}

\index{maneuverable (fspsim.utils.SpaceObject.SpaceObject attribute)@\spxentry{maneuverable}\spxextra{fspsim.utils.SpaceObject.SpaceObject attribute}}

\begin{fulllineitems}
\phantomsection\label{\detokenize{fspsim.utils:fspsim.utils.SpaceObject.SpaceObject.maneuverable}}
\pysigstartsignatures
\pysigline{\sphinxbfcode{\sphinxupquote{maneuverable}}}
\pysigstopsignatures
\sphinxAtStartPar
Indicates if the object is maneuverable.
\begin{quote}\begin{description}
\sphinxlineitem{Type}
\sphinxAtStartPar
str

\end{description}\end{quote}

\end{fulllineitems}

\index{spin\_stabilized (fspsim.utils.SpaceObject.SpaceObject attribute)@\spxentry{spin\_stabilized}\spxextra{fspsim.utils.SpaceObject.SpaceObject attribute}}

\begin{fulllineitems}
\phantomsection\label{\detokenize{fspsim.utils:fspsim.utils.SpaceObject.SpaceObject.spin_stabilized}}
\pysigstartsignatures
\pysigline{\sphinxbfcode{\sphinxupquote{spin\_stabilized}}}
\pysigstopsignatures
\sphinxAtStartPar
Indicates if the object is spin\sphinxhyphen{}stabilized.
\begin{quote}\begin{description}
\sphinxlineitem{Type}
\sphinxAtStartPar
str

\end{description}\end{quote}

\end{fulllineitems}

\index{apogee (fspsim.utils.SpaceObject.SpaceObject attribute)@\spxentry{apogee}\spxextra{fspsim.utils.SpaceObject.SpaceObject attribute}}

\begin{fulllineitems}
\phantomsection\label{\detokenize{fspsim.utils:fspsim.utils.SpaceObject.SpaceObject.apogee}}
\pysigstartsignatures
\pysigline{\sphinxbfcode{\sphinxupquote{apogee}}}
\pysigstopsignatures
\sphinxAtStartPar
The maximum altitude of the object’s orbit.
\begin{quote}\begin{description}
\sphinxlineitem{Type}
\sphinxAtStartPar
float

\end{description}\end{quote}

\end{fulllineitems}

\index{perigee (fspsim.utils.SpaceObject.SpaceObject attribute)@\spxentry{perigee}\spxextra{fspsim.utils.SpaceObject.SpaceObject attribute}}

\begin{fulllineitems}
\phantomsection\label{\detokenize{fspsim.utils:fspsim.utils.SpaceObject.SpaceObject.perigee}}
\pysigstartsignatures
\pysigline{\sphinxbfcode{\sphinxupquote{perigee}}}
\pysigstopsignatures
\sphinxAtStartPar
The minimum altitude of the object’s orbit.
\begin{quote}\begin{description}
\sphinxlineitem{Type}
\sphinxAtStartPar
float

\end{description}\end{quote}

\end{fulllineitems}

\index{propulsion\_type (fspsim.utils.SpaceObject.SpaceObject attribute)@\spxentry{propulsion\_type}\spxextra{fspsim.utils.SpaceObject.SpaceObject attribute}}

\begin{fulllineitems}
\phantomsection\label{\detokenize{fspsim.utils:fspsim.utils.SpaceObject.SpaceObject.propulsion_type}}
\pysigstartsignatures
\pysigline{\sphinxbfcode{\sphinxupquote{propulsion\_type}}}
\pysigstopsignatures
\sphinxAtStartPar
Type of propulsion used by the object.
\begin{quote}\begin{description}
\sphinxlineitem{Type}
\sphinxAtStartPar
str

\end{description}\end{quote}

\end{fulllineitems}

\index{epoch (fspsim.utils.SpaceObject.SpaceObject attribute)@\spxentry{epoch}\spxextra{fspsim.utils.SpaceObject.SpaceObject attribute}}

\begin{fulllineitems}
\phantomsection\label{\detokenize{fspsim.utils:fspsim.utils.SpaceObject.SpaceObject.epoch}}
\pysigstartsignatures
\pysigline{\sphinxbfcode{\sphinxupquote{epoch}}}
\pysigstopsignatures
\sphinxAtStartPar
Reference time for the object’s orbital parameters.
\begin{quote}\begin{description}
\sphinxlineitem{Type}
\sphinxAtStartPar
datetime

\end{description}\end{quote}

\end{fulllineitems}

\index{day\_of\_year (fspsim.utils.SpaceObject.SpaceObject attribute)@\spxentry{day\_of\_year}\spxextra{fspsim.utils.SpaceObject.SpaceObject attribute}}

\begin{fulllineitems}
\phantomsection\label{\detokenize{fspsim.utils:fspsim.utils.SpaceObject.SpaceObject.day_of_year}}
\pysigstartsignatures
\pysigline{\sphinxbfcode{\sphinxupquote{day\_of\_year}}}
\pysigstopsignatures
\sphinxAtStartPar
Day of the year derived from the epoch.
\begin{quote}\begin{description}
\sphinxlineitem{Type}
\sphinxAtStartPar
int

\end{description}\end{quote}

\end{fulllineitems}

\index{station\_keeping (fspsim.utils.SpaceObject.SpaceObject attribute)@\spxentry{station\_keeping}\spxextra{fspsim.utils.SpaceObject.SpaceObject attribute}}

\begin{fulllineitems}
\phantomsection\label{\detokenize{fspsim.utils:fspsim.utils.SpaceObject.SpaceObject.station_keeping}}
\pysigstartsignatures
\pysigline{\sphinxbfcode{\sphinxupquote{station\_keeping}}}
\pysigstopsignatures
\sphinxAtStartPar
Dates indicating station keeping period.
\begin{quote}\begin{description}
\sphinxlineitem{Type}
\sphinxAtStartPar
list or bool or None

\end{description}\end{quote}

\end{fulllineitems}

\index{ephemeris (fspsim.utils.SpaceObject.SpaceObject attribute)@\spxentry{ephemeris}\spxextra{fspsim.utils.SpaceObject.SpaceObject attribute}}

\begin{fulllineitems}
\phantomsection\label{\detokenize{fspsim.utils:fspsim.utils.SpaceObject.SpaceObject.ephemeris}}
\pysigstartsignatures
\pysigline{\sphinxbfcode{\sphinxupquote{ephemeris}}}
\pysigstopsignatures
\sphinxAtStartPar
Ephemeris data for the object.
\begin{quote}\begin{description}
\sphinxlineitem{Type}
\sphinxAtStartPar
list

\end{description}\end{quote}

\end{fulllineitems}

\index{sma (fspsim.utils.SpaceObject.SpaceObject attribute)@\spxentry{sma}\spxextra{fspsim.utils.SpaceObject.SpaceObject attribute}}

\begin{fulllineitems}
\phantomsection\label{\detokenize{fspsim.utils:fspsim.utils.SpaceObject.SpaceObject.sma}}
\pysigstartsignatures
\pysigline{\sphinxbfcode{\sphinxupquote{sma}}}
\pysigstopsignatures
\sphinxAtStartPar
Semi\sphinxhyphen{}major axis of the object’s orbit.
\begin{quote}\begin{description}
\sphinxlineitem{Type}
\sphinxAtStartPar
float

\end{description}\end{quote}

\end{fulllineitems}

\index{orbital\_period (fspsim.utils.SpaceObject.SpaceObject attribute)@\spxentry{orbital\_period}\spxextra{fspsim.utils.SpaceObject.SpaceObject attribute}}

\begin{fulllineitems}
\phantomsection\label{\detokenize{fspsim.utils:fspsim.utils.SpaceObject.SpaceObject.orbital_period}}
\pysigstartsignatures
\pysigline{\sphinxbfcode{\sphinxupquote{orbital\_period}}}
\pysigstopsignatures
\sphinxAtStartPar
Orbital period of the object.
\begin{quote}\begin{description}
\sphinxlineitem{Type}
\sphinxAtStartPar
float

\end{description}\end{quote}

\end{fulllineitems}

\index{inc (fspsim.utils.SpaceObject.SpaceObject attribute)@\spxentry{inc}\spxextra{fspsim.utils.SpaceObject.SpaceObject attribute}}

\begin{fulllineitems}
\phantomsection\label{\detokenize{fspsim.utils:fspsim.utils.SpaceObject.SpaceObject.inc}}
\pysigstartsignatures
\pysigline{\sphinxbfcode{\sphinxupquote{inc}}}
\pysigstopsignatures
\sphinxAtStartPar
Inclination of the object’s orbit.
\begin{quote}\begin{description}
\sphinxlineitem{Type}
\sphinxAtStartPar
float

\end{description}\end{quote}

\end{fulllineitems}

\index{argp (fspsim.utils.SpaceObject.SpaceObject attribute)@\spxentry{argp}\spxextra{fspsim.utils.SpaceObject.SpaceObject attribute}}

\begin{fulllineitems}
\phantomsection\label{\detokenize{fspsim.utils:fspsim.utils.SpaceObject.SpaceObject.argp}}
\pysigstartsignatures
\pysigline{\sphinxbfcode{\sphinxupquote{argp}}}
\pysigstopsignatures
\sphinxAtStartPar
Argument of perigee.
\begin{quote}\begin{description}
\sphinxlineitem{Type}
\sphinxAtStartPar
float

\end{description}\end{quote}

\end{fulllineitems}

\index{raan (fspsim.utils.SpaceObject.SpaceObject attribute)@\spxentry{raan}\spxextra{fspsim.utils.SpaceObject.SpaceObject attribute}}

\begin{fulllineitems}
\phantomsection\label{\detokenize{fspsim.utils:fspsim.utils.SpaceObject.SpaceObject.raan}}
\pysigstartsignatures
\pysigline{\sphinxbfcode{\sphinxupquote{raan}}}
\pysigstopsignatures
\sphinxAtStartPar
Right ascension of the ascending node.
\begin{quote}\begin{description}
\sphinxlineitem{Type}
\sphinxAtStartPar
float

\end{description}\end{quote}

\end{fulllineitems}

\index{tran (fspsim.utils.SpaceObject.SpaceObject attribute)@\spxentry{tran}\spxextra{fspsim.utils.SpaceObject.SpaceObject attribute}}

\begin{fulllineitems}
\phantomsection\label{\detokenize{fspsim.utils:fspsim.utils.SpaceObject.SpaceObject.tran}}
\pysigstartsignatures
\pysigline{\sphinxbfcode{\sphinxupquote{tran}}}
\pysigstopsignatures
\sphinxAtStartPar
True anomaly.
\begin{quote}\begin{description}
\sphinxlineitem{Type}
\sphinxAtStartPar
float

\end{description}\end{quote}

\end{fulllineitems}

\index{eccentricity (fspsim.utils.SpaceObject.SpaceObject attribute)@\spxentry{eccentricity}\spxextra{fspsim.utils.SpaceObject.SpaceObject attribute}}

\begin{fulllineitems}
\phantomsection\label{\detokenize{fspsim.utils:fspsim.utils.SpaceObject.SpaceObject.eccentricity}}
\pysigstartsignatures
\pysigline{\sphinxbfcode{\sphinxupquote{eccentricity}}}
\pysigstopsignatures
\sphinxAtStartPar
Eccentricity of the object’s orbit.
\begin{quote}\begin{description}
\sphinxlineitem{Type}
\sphinxAtStartPar
float

\end{description}\end{quote}

\end{fulllineitems}

\index{meananomaly (fspsim.utils.SpaceObject.SpaceObject attribute)@\spxentry{meananomaly}\spxextra{fspsim.utils.SpaceObject.SpaceObject attribute}}

\begin{fulllineitems}
\phantomsection\label{\detokenize{fspsim.utils:fspsim.utils.SpaceObject.SpaceObject.meananomaly}}
\pysigstartsignatures
\pysigline{\sphinxbfcode{\sphinxupquote{meananomaly}}}
\pysigstopsignatures
\sphinxAtStartPar
Mean anomaly.
\begin{quote}\begin{description}
\sphinxlineitem{Type}
\sphinxAtStartPar
float

\end{description}\end{quote}

\end{fulllineitems}

\index{cart\_state (fspsim.utils.SpaceObject.SpaceObject attribute)@\spxentry{cart\_state}\spxextra{fspsim.utils.SpaceObject.SpaceObject attribute}}

\begin{fulllineitems}
\phantomsection\label{\detokenize{fspsim.utils:fspsim.utils.SpaceObject.SpaceObject.cart_state}}
\pysigstartsignatures
\pysigline{\sphinxbfcode{\sphinxupquote{cart\_state}}}
\pysigstopsignatures
\sphinxAtStartPar
Cartesian state vector {[}x,y,z,u,v,w{]}.
\begin{quote}\begin{description}
\sphinxlineitem{Type}
\sphinxAtStartPar
np.array

\end{description}\end{quote}

\end{fulllineitems}

\index{C\_d (fspsim.utils.SpaceObject.SpaceObject attribute)@\spxentry{C\_d}\spxextra{fspsim.utils.SpaceObject.SpaceObject attribute}}

\begin{fulllineitems}
\phantomsection\label{\detokenize{fspsim.utils:fspsim.utils.SpaceObject.SpaceObject.C_d}}
\pysigstartsignatures
\pysigline{\sphinxbfcode{\sphinxupquote{C\_d}}}
\pysigstopsignatures
\sphinxAtStartPar
Drag coefficient.
\begin{quote}\begin{description}
\sphinxlineitem{Type}
\sphinxAtStartPar
float

\end{description}\end{quote}

\end{fulllineitems}

\index{tle (fspsim.utils.SpaceObject.SpaceObject attribute)@\spxentry{tle}\spxextra{fspsim.utils.SpaceObject.SpaceObject attribute}}

\begin{fulllineitems}
\phantomsection\label{\detokenize{fspsim.utils:fspsim.utils.SpaceObject.SpaceObject.tle}}
\pysigstartsignatures
\pysigline{\sphinxbfcode{\sphinxupquote{tle}}}
\pysigstopsignatures
\sphinxAtStartPar
Two\sphinxhyphen{}line element set representing the object’s orbit.
\begin{quote}\begin{description}
\sphinxlineitem{Type}
\sphinxAtStartPar
str

\end{description}\end{quote}

\end{fulllineitems}

\index{generate\_cart() (fspsim.utils.SpaceObject.SpaceObject method)@\spxentry{generate\_cart()}\spxextra{fspsim.utils.SpaceObject.SpaceObject method}}

\begin{fulllineitems}
\phantomsection\label{\detokenize{fspsim.utils:fspsim.utils.SpaceObject.SpaceObject.generate_cart}}
\pysigstartsignatures
\pysiglinewithargsret{\sphinxbfcode{\sphinxupquote{generate\_cart}}}{}{}
\pysigstopsignatures
\sphinxAtStartPar
Generates a cartesian state vector from keplerian elements.
\begin{quote}\begin{description}
\sphinxlineitem{Returns}
\sphinxAtStartPar
Cartesian state.

\sphinxlineitem{Return type}
\sphinxAtStartPar
np.array

\end{description}\end{quote}

\end{fulllineitems}

\index{impute\_char\_area() (fspsim.utils.SpaceObject.SpaceObject method)@\spxentry{impute\_char\_area()}\spxextra{fspsim.utils.SpaceObject.SpaceObject method}}

\begin{fulllineitems}
\phantomsection\label{\detokenize{fspsim.utils:fspsim.utils.SpaceObject.SpaceObject.impute_char_area}}
\pysigstartsignatures
\pysiglinewithargsret{\sphinxbfcode{\sphinxupquote{impute\_char\_area}}}{\sphinxparam{\DUrole{n}{char\_area}}}{}
\pysigstopsignatures
\sphinxAtStartPar
Imputes a characteristic area based on the object type if the given value is None or 0.
\begin{quote}\begin{description}
\sphinxlineitem{Parameters}
\sphinxAtStartPar
\sphinxstyleliteralstrong{\sphinxupquote{char\_area}} (\sphinxstyleliteralemphasis{\sphinxupquote{float}}\sphinxstyleliteralemphasis{\sphinxupquote{ or }}\sphinxstyleliteralemphasis{\sphinxupquote{None}}) – Characteristic area to verify or impute.

\sphinxlineitem{Returns}
\sphinxAtStartPar
Imputed or verified characteristic area.

\sphinxlineitem{Return type}
\sphinxAtStartPar
float

\sphinxlineitem{Raises}
\sphinxAtStartPar
\sphinxstyleliteralstrong{\sphinxupquote{ValueError}} – If the imputed or given char\_area is None or 0.

\end{description}\end{quote}

\end{fulllineitems}

\index{impute\_char\_length() (fspsim.utils.SpaceObject.SpaceObject method)@\spxentry{impute\_char\_length()}\spxextra{fspsim.utils.SpaceObject.SpaceObject method}}

\begin{fulllineitems}
\phantomsection\label{\detokenize{fspsim.utils:fspsim.utils.SpaceObject.SpaceObject.impute_char_length}}
\pysigstartsignatures
\pysiglinewithargsret{\sphinxbfcode{\sphinxupquote{impute\_char\_length}}}{\sphinxparam{\DUrole{n}{char\_length}}}{}
\pysigstopsignatures
\sphinxAtStartPar
Imputes a characteristic length based on the object type if the given value is None or 0.
\begin{quote}\begin{description}
\sphinxlineitem{Parameters}
\sphinxAtStartPar
\sphinxstyleliteralstrong{\sphinxupquote{char\_length}} (\sphinxstyleliteralemphasis{\sphinxupquote{float}}\sphinxstyleliteralemphasis{\sphinxupquote{ or }}\sphinxstyleliteralemphasis{\sphinxupquote{None}}) – Characteristic length to verify or impute.

\sphinxlineitem{Returns}
\sphinxAtStartPar
Imputed or verified characteristic length.

\sphinxlineitem{Return type}
\sphinxAtStartPar
float

\sphinxlineitem{Raises}
\sphinxAtStartPar
\sphinxstyleliteralstrong{\sphinxupquote{ValueError}} – If the imputed or given char\_length is None or 0.

\end{description}\end{quote}

\end{fulllineitems}

\index{impute\_mass() (fspsim.utils.SpaceObject.SpaceObject method)@\spxentry{impute\_mass()}\spxextra{fspsim.utils.SpaceObject.SpaceObject method}}

\begin{fulllineitems}
\phantomsection\label{\detokenize{fspsim.utils:fspsim.utils.SpaceObject.SpaceObject.impute_mass}}
\pysigstartsignatures
\pysiglinewithargsret{\sphinxbfcode{\sphinxupquote{impute\_mass}}}{\sphinxparam{\DUrole{n}{mass}}}{}
\pysigstopsignatures
\sphinxAtStartPar
Imputes a mass based on the object characteristic length if the given value is None or 0.
The relationship between mass and characteristic length is based on the relationship between mass and length reported in From Alfano, Oltrogge and Sheppard, 2020
\begin{quote}\begin{description}
\sphinxlineitem{Parameters}
\sphinxAtStartPar
\sphinxstyleliteralstrong{\sphinxupquote{mass}} (\sphinxstyleliteralemphasis{\sphinxupquote{float}}\sphinxstyleliteralemphasis{\sphinxupquote{ or }}\sphinxstyleliteralemphasis{\sphinxupquote{None}}\sphinxstyleliteralemphasis{\sphinxupquote{ or }}\sphinxstyleliteralemphasis{\sphinxupquote{str}}) – Mass to verify or impute.

\sphinxlineitem{Returns}
\sphinxAtStartPar
Imputed mass value.

\sphinxlineitem{Return type}
\sphinxAtStartPar
float

\end{description}\end{quote}

\end{fulllineitems}

\index{prop\_catobject() (fspsim.utils.SpaceObject.SpaceObject method)@\spxentry{prop\_catobject()}\spxextra{fspsim.utils.SpaceObject.SpaceObject method}}

\begin{fulllineitems}
\phantomsection\label{\detokenize{fspsim.utils:fspsim.utils.SpaceObject.SpaceObject.prop_catobject}}
\pysigstartsignatures
\pysiglinewithargsret{\sphinxbfcode{\sphinxupquote{prop\_catobject}}}{\sphinxparam{\DUrole{n}{jd\_start}}\sphinxparamcomma \sphinxparam{\DUrole{n}{jd\_stop}}\sphinxparamcomma \sphinxparam{\DUrole{n}{step\_size}}\sphinxparamcomma \sphinxparam{\DUrole{n}{output\_freq}}}{}
\pysigstopsignatures
\sphinxAtStartPar
Propagates the object based on initial conditions, propagator type, and station keeping preferences.
\begin{quote}\begin{description}
\sphinxlineitem{Parameters}\begin{itemize}
\item {} 
\sphinxAtStartPar
\sphinxstyleliteralstrong{\sphinxupquote{jd\_start}} (\sphinxstyleliteralemphasis{\sphinxupquote{float}}) – Julian start date for the simulation.

\item {} 
\sphinxAtStartPar
\sphinxstyleliteralstrong{\sphinxupquote{jd\_stop}} (\sphinxstyleliteralemphasis{\sphinxupquote{float}}) – Julian stop date for simulation.

\item {} 
\sphinxAtStartPar
\sphinxstyleliteralstrong{\sphinxupquote{step\_size}} (\sphinxstyleliteralemphasis{\sphinxupquote{float}}) – Step size for propagation.

\item {} 
\sphinxAtStartPar
\sphinxstyleliteralstrong{\sphinxupquote{output\_freq}} (\sphinxstyleliteralemphasis{\sphinxupquote{float}}) – Frequency at which to output the ephemeris (in seconds).

\item {} 
\sphinxAtStartPar
\sphinxstyleliteralstrong{\sphinxupquote{integrator\_type}} (\sphinxstyleliteralemphasis{\sphinxupquote{str}}) – Numerical integrator to use.

\item {} 
\sphinxAtStartPar
\sphinxstyleliteralstrong{\sphinxupquote{use\_sgp4\_propagation}} (\sphinxstyleliteralemphasis{\sphinxupquote{bool}}) – Propagate using SGP4 for 100\sphinxhyphen{}minute segments.

\end{itemize}

\sphinxlineitem{Returns}
\sphinxAtStartPar
None. Updates the \sphinxtitleref{ephemeris} attribute of the object.

\end{description}\end{quote}

\end{fulllineitems}


\end{fulllineitems}

\index{verify\_angle() (in module fspsim.utils.SpaceObject)@\spxentry{verify\_angle()}\spxextra{in module fspsim.utils.SpaceObject}}

\begin{fulllineitems}
\phantomsection\label{\detokenize{fspsim.utils:fspsim.utils.SpaceObject.verify_angle}}
\pysigstartsignatures
\pysiglinewithargsret{\sphinxcode{\sphinxupquote{fspsim.utils.SpaceObject.}}\sphinxbfcode{\sphinxupquote{verify\_angle}}}{\sphinxparam{\DUrole{n}{value}}\sphinxparamcomma \sphinxparam{\DUrole{n}{name}}\sphinxparamcomma \sphinxparam{\DUrole{n}{random}\DUrole{o}{=}\DUrole{default_value}{False}}}{}
\pysigstopsignatures
\sphinxAtStartPar
Verifies that the value is a float and is between 0 and 360. If it is None or not
a float or out of range, raise a ValueError.
\begin{quote}\begin{description}
\sphinxlineitem{Parameters}\begin{itemize}
\item {} 
\sphinxAtStartPar
\sphinxstyleliteralstrong{\sphinxupquote{value}} (\sphinxstyleliteralemphasis{\sphinxupquote{float}}\sphinxstyleliteralemphasis{\sphinxupquote{ or }}\sphinxstyleliteralemphasis{\sphinxupquote{None}}\sphinxstyleliteralemphasis{\sphinxupquote{ or }}\sphinxstyleliteralemphasis{\sphinxupquote{str}}) – The value to verify

\item {} 
\sphinxAtStartPar
\sphinxstyleliteralstrong{\sphinxupquote{name}} (\sphinxstyleliteralemphasis{\sphinxupquote{str}}) – Name of the object

\item {} 
\sphinxAtStartPar
\sphinxstyleliteralstrong{\sphinxupquote{random}} (\sphinxstyleliteralemphasis{\sphinxupquote{bool}}\sphinxstyleliteralemphasis{\sphinxupquote{, }}\sphinxstyleliteralemphasis{\sphinxupquote{optional}}) – If \sphinxtitleref{True}, return a random angle between 0 and 360 for invalid values, defaults to \sphinxtitleref{False}

\end{itemize}

\sphinxlineitem{Returns}
\sphinxAtStartPar
The verified angle or a random angle if \sphinxtitleref{random} is \sphinxtitleref{True}

\sphinxlineitem{Return type}
\sphinxAtStartPar
float

\sphinxlineitem{Raises}
\sphinxAtStartPar
\sphinxstyleliteralstrong{\sphinxupquote{ValueError}} – If the value is not a float or out of range and \sphinxtitleref{random} is \sphinxtitleref{False}

\end{description}\end{quote}

\end{fulllineitems}

\index{verify\_eccentricity() (in module fspsim.utils.SpaceObject)@\spxentry{verify\_eccentricity()}\spxextra{in module fspsim.utils.SpaceObject}}

\begin{fulllineitems}
\phantomsection\label{\detokenize{fspsim.utils:fspsim.utils.SpaceObject.verify_eccentricity}}
\pysigstartsignatures
\pysiglinewithargsret{\sphinxcode{\sphinxupquote{fspsim.utils.SpaceObject.}}\sphinxbfcode{\sphinxupquote{verify\_eccentricity}}}{\sphinxparam{\DUrole{n}{value}}}{}
\pysigstopsignatures
\sphinxAtStartPar
Verifies that the value is a float and is between 0 and 1. If it is None or not a
float or out of range, raise a ValueError.
\begin{quote}\begin{description}
\sphinxlineitem{Parameters}
\sphinxAtStartPar
\sphinxstyleliteralstrong{\sphinxupquote{value}} (\sphinxstyleliteralemphasis{\sphinxupquote{float}}\sphinxstyleliteralemphasis{\sphinxupquote{ or }}\sphinxstyleliteralemphasis{\sphinxupquote{None}}\sphinxstyleliteralemphasis{\sphinxupquote{ or }}\sphinxstyleliteralemphasis{\sphinxupquote{str}}) – The value to verify

\sphinxlineitem{Returns}
\sphinxAtStartPar
The verified value

\sphinxlineitem{Return type}
\sphinxAtStartPar
float

\sphinxlineitem{Raises}
\sphinxAtStartPar
\sphinxstyleliteralstrong{\sphinxupquote{ValueError}} – If the value is not a float or out of range

\end{description}\end{quote}

\end{fulllineitems}

\index{verify\_value() (in module fspsim.utils.SpaceObject)@\spxentry{verify\_value()}\spxextra{in module fspsim.utils.SpaceObject}}

\begin{fulllineitems}
\phantomsection\label{\detokenize{fspsim.utils:fspsim.utils.SpaceObject.verify_value}}
\pysigstartsignatures
\pysiglinewithargsret{\sphinxcode{\sphinxupquote{fspsim.utils.SpaceObject.}}\sphinxbfcode{\sphinxupquote{verify\_value}}}{\sphinxparam{\DUrole{n}{value}}\sphinxparamcomma \sphinxparam{\DUrole{n}{impute\_function}}}{}
\pysigstopsignatures
\sphinxAtStartPar
Verifies that the value is a float and is not None. If it is None or not a float or
too small, then impute the value using the impute\_function.
\begin{quote}\begin{description}
\sphinxlineitem{Parameters}\begin{itemize}
\item {} 
\sphinxAtStartPar
\sphinxstyleliteralstrong{\sphinxupquote{value}} (\sphinxstyleliteralemphasis{\sphinxupquote{float}}\sphinxstyleliteralemphasis{\sphinxupquote{ or }}\sphinxstyleliteralemphasis{\sphinxupquote{None}}\sphinxstyleliteralemphasis{\sphinxupquote{ or }}\sphinxstyleliteralemphasis{\sphinxupquote{str}}) – The value to verify

\item {} 
\sphinxAtStartPar
\sphinxstyleliteralstrong{\sphinxupquote{impute\_function}} (\sphinxstyleliteralemphasis{\sphinxupquote{function}}) – Function to impute value if necessary

\end{itemize}

\sphinxlineitem{Returns}
\sphinxAtStartPar
The verified or imputed value

\sphinxlineitem{Return type}
\sphinxAtStartPar
float

\end{description}\end{quote}

\end{fulllineitems}



\paragraph{Module contents}
\label{\detokenize{fspsim.utils:module-fspsim.utils}}\label{\detokenize{fspsim.utils:module-contents}}\index{module@\spxentry{module}!fspsim.utils@\spxentry{fspsim.utils}}\index{fspsim.utils@\spxentry{fspsim.utils}!module@\spxentry{module}}

\subsection{Submodules}
\label{\detokenize{fspsim:submodules}}

\subsection{fspsim.simulate module}
\label{\detokenize{fspsim:module-fspsim.simulate}}\label{\detokenize{fspsim:fspsim-simulate-module}}\index{module@\spxentry{module}!fspsim.simulate@\spxentry{fspsim.simulate}}\index{fspsim.simulate@\spxentry{fspsim.simulate}!module@\spxentry{module}}\index{dump\_pickle() (in module fspsim.simulate)@\spxentry{dump\_pickle()}\spxextra{in module fspsim.simulate}}

\begin{fulllineitems}
\phantomsection\label{\detokenize{fspsim:fspsim.simulate.dump_pickle}}
\pysigstartsignatures
\pysiglinewithargsret{\sphinxcode{\sphinxupquote{fspsim.simulate.}}\sphinxbfcode{\sphinxupquote{dump\_pickle}}}{\sphinxparam{\DUrole{n}{file\_path}}\sphinxparamcomma \sphinxparam{\DUrole{n}{data}}}{}
\pysigstopsignatures
\end{fulllineitems}

\index{get\_path() (in module fspsim.simulate)@\spxentry{get\_path()}\spxextra{in module fspsim.simulate}}

\begin{fulllineitems}
\phantomsection\label{\detokenize{fspsim:fspsim.simulate.get_path}}
\pysigstartsignatures
\pysiglinewithargsret{\sphinxcode{\sphinxupquote{fspsim.simulate.}}\sphinxbfcode{\sphinxupquote{get\_path}}}{\sphinxparam{\DUrole{o}{*}\DUrole{n}{args}}}{}
\pysigstopsignatures
\end{fulllineitems}

\index{load\_pickle() (in module fspsim.simulate)@\spxentry{load\_pickle()}\spxextra{in module fspsim.simulate}}

\begin{fulllineitems}
\phantomsection\label{\detokenize{fspsim:fspsim.simulate.load_pickle}}
\pysigstartsignatures
\pysiglinewithargsret{\sphinxcode{\sphinxupquote{fspsim.simulate.}}\sphinxbfcode{\sphinxupquote{load\_pickle}}}{\sphinxparam{\DUrole{n}{file\_path}}}{}
\pysigstopsignatures
\end{fulllineitems}

\index{propagate\_space\_object() (in module fspsim.simulate)@\spxentry{propagate\_space\_object()}\spxextra{in module fspsim.simulate}}

\begin{fulllineitems}
\phantomsection\label{\detokenize{fspsim:fspsim.simulate.propagate_space_object}}
\pysigstartsignatures
\pysiglinewithargsret{\sphinxcode{\sphinxupquote{fspsim.simulate.}}\sphinxbfcode{\sphinxupquote{propagate\_space\_object}}}{\sphinxparam{\DUrole{n}{args}}}{}
\pysigstopsignatures
\end{fulllineitems}

\index{run\_sim() (in module fspsim.simulate)@\spxentry{run\_sim()}\spxextra{in module fspsim.simulate}}

\begin{fulllineitems}
\phantomsection\label{\detokenize{fspsim:fspsim.simulate.run_sim}}
\pysigstartsignatures
\pysiglinewithargsret{\sphinxcode{\sphinxupquote{fspsim.simulate.}}\sphinxbfcode{\sphinxupquote{run\_sim}}}{\sphinxparam{\DUrole{n}{settings: <module 'json' from '/Users/charlesc/anaconda3/lib/python3.11/json/\_\_init\_\_.py'>}}\sphinxparamcomma \sphinxparam{\DUrole{n}{future\_constellations\_file: str = None}}}{{ $\rightarrow$ None}}
\pysigstopsignatures
\sphinxAtStartPar
Propagates a list of space objects over a specified time range.

\sphinxAtStartPar
The function retrieves space objects from a space catalogue (SATCAT), propagates each object
up to a the specified time. This will save the output locally but will also return a list of propagated
space ojects. A progress bar is displayed to track the propagation process.
\begin{quote}\begin{description}
\sphinxlineitem{Parameters}\begin{itemize}
\item {} 
\sphinxAtStartPar
\sphinxstyleliteralstrong{\sphinxupquote{settings}} (\sphinxstyleliteralemphasis{\sphinxupquote{\sphinxhyphen{}}}) – A dictionary containing the following key\sphinxhyphen{}value pairs:

\item {} 
\sphinxAtStartPar
\sphinxstyleliteralstrong{\sphinxupquote{"sim\_start\_date"}} (\sphinxstyleliteralemphasis{\sphinxupquote{\sphinxhyphen{}}}) – UTC start date for simulation (str)

\item {} 
\sphinxAtStartPar
\sphinxstyleliteralstrong{\sphinxupquote{"sim\_end\_date"}} (\sphinxstyleliteralemphasis{\sphinxupquote{\sphinxhyphen{}}}) – UTC end date for simulation (str)

\item {} 
\sphinxAtStartPar
\sphinxstyleliteralstrong{\sphinxupquote{"integrator\_step\_size"}} (\sphinxstyleliteralemphasis{\sphinxupquote{\sphinxhyphen{}}}) – Time step size for integrator (str, converted to int)

\item {} 
\sphinxAtStartPar
\sphinxstyleliteralstrong{\sphinxupquote{"output\_frequency"}} (\sphinxstyleliteralemphasis{\sphinxupquote{\sphinxhyphen{}}}) – Output frequency (str, converted to int)

\item {} 
\sphinxAtStartPar
\sphinxstyleliteralstrong{\sphinxupquote{"scenario\_name"}} (\sphinxstyleliteralemphasis{\sphinxupquote{\sphinxhyphen{}}}) – Name of the scenario (str)

\item {} 
\sphinxAtStartPar
\sphinxstyleliteralstrong{\sphinxupquote{"integrator\_type"}} (\sphinxstyleliteralemphasis{\sphinxupquote{\sphinxhyphen{}}}) – Type of integrator to use (str)

\item {} 
\sphinxAtStartPar
\sphinxstyleliteralstrong{\sphinxupquote{"sgp4\_long\_term"}} (\sphinxstyleliteralemphasis{\sphinxupquote{\sphinxhyphen{}}}) – Boolean indicating if SGP4 long term propagation should be used (str, converted to bool)

\item {} 
\sphinxAtStartPar
\sphinxstyleliteralstrong{\sphinxupquote{"force\_model"}} (\sphinxstyleliteralemphasis{\sphinxupquote{\sphinxhyphen{}}}) – Force model settings (can be various types, depending on implementation)

\end{itemize}

\end{description}\end{quote}

\sphinxAtStartPar
Returns:
SATCAT Catalogue with updated ephemeris of the locations.
Results are saved to pickle files in the directory: ‘src/fspsim/data/results/propagated\_catalogs/’.

\sphinxAtStartPar
Notes:
\sphinxhyphen{} Each batch consists of a set number of space objects (defined by the batch\_size variable).
\sphinxhyphen{} The pickle files are saved with a filename pattern: ‘<scenario\_name>\_batch\_<batch\_number>.pickle’.
\sphinxhyphen{} The propagation function used is ‘propagate\_space\_object’ (not defined in the provided code snippet).

\end{fulllineitems}

\index{set\_future\_constellations() (in module fspsim.simulate)@\spxentry{set\_future\_constellations()}\spxextra{in module fspsim.simulate}}

\begin{fulllineitems}
\phantomsection\label{\detokenize{fspsim:fspsim.simulate.set_future_constellations}}
\pysigstartsignatures
\pysiglinewithargsret{\sphinxcode{\sphinxupquote{fspsim.simulate.}}\sphinxbfcode{\sphinxupquote{set\_future\_constellations}}}{\sphinxparam{\DUrole{n}{constellations}}}{{ $\rightarrow$ bool}}
\pysigstopsignatures
\sphinxAtStartPar
Incorporates user\sphinxhyphen{}specified launch predictions into the simulation.
This will check that it is in the correct format
and can be used by the simulation.
\begin{quote}\begin{description}
\sphinxlineitem{Parameters}
\sphinxAtStartPar
\sphinxstyleliteralstrong{\sphinxupquote{constellations}} (\sphinxstyleliteralemphasis{\sphinxupquote{str}}) – The path of the specified csv file

\sphinxlineitem{Returns}
\sphinxAtStartPar
If it is in the correct format.

\sphinxlineitem{Return type}
\sphinxAtStartPar
bool

\end{description}\end{quote}

\end{fulllineitems}



\subsection{Module contents}
\label{\detokenize{fspsim:module-fspsim}}\label{\detokenize{fspsim:module-contents}}\index{module@\spxentry{module}!fspsim@\spxentry{fspsim}}\index{fspsim@\spxentry{fspsim}!module@\spxentry{module}}

\chapter{Indices and tables}
\label{\detokenize{index:indices-and-tables}}\begin{itemize}
\item {} 
\sphinxAtStartPar
\DUrole{xref,std,std-ref}{genindex}

\item {} 
\sphinxAtStartPar
\DUrole{xref,std,std-ref}{modindex}

\item {} 
\sphinxAtStartPar
\DUrole{xref,std,std-ref}{search}

\end{itemize}


\renewcommand{\indexname}{Python Module Index}
\begin{sphinxtheindex}
\let\bigletter\sphinxstyleindexlettergroup
\bigletter{f}
\item\relax\sphinxstyleindexentry{fspsim}\sphinxstyleindexpageref{fspsim:\detokenize{module-fspsim}}
\item\relax\sphinxstyleindexentry{fspsim.simulate}\sphinxstyleindexpageref{fspsim:\detokenize{module-fspsim.simulate}}
\item\relax\sphinxstyleindexentry{fspsim.utils}\sphinxstyleindexpageref{fspsim.utils:\detokenize{module-fspsim.utils}}
\item\relax\sphinxstyleindexentry{fspsim.utils.Conversions}\sphinxstyleindexpageref{fspsim.utils:\detokenize{module-fspsim.utils.Conversions}}
\item\relax\sphinxstyleindexentry{fspsim.utils.Formatting}\sphinxstyleindexpageref{fspsim.utils:\detokenize{module-fspsim.utils.Formatting}}
\item\relax\sphinxstyleindexentry{fspsim.utils.LaunchModel}\sphinxstyleindexpageref{fspsim.utils:\detokenize{module-fspsim.utils.LaunchModel}}
\item\relax\sphinxstyleindexentry{fspsim.utils.Propagators}\sphinxstyleindexpageref{fspsim.utils:\detokenize{module-fspsim.utils.Propagators}}
\item\relax\sphinxstyleindexentry{fspsim.utils.SpaceCatalogue}\sphinxstyleindexpageref{fspsim.utils:\detokenize{module-fspsim.utils.SpaceCatalogue}}
\item\relax\sphinxstyleindexentry{fspsim.utils.SpaceObject}\sphinxstyleindexpageref{fspsim.utils:\detokenize{module-fspsim.utils.SpaceObject}}
\end{sphinxtheindex}

\renewcommand{\indexname}{Index}
\printindex
\end{document}